\documentclass{styles/uscthesis}

%%%%%%%%%%%%%%%%%%%%%%%%%%%%%%%%%%%%%%%%%%
%%% Options include: [forbinding], which produces 
%%% an alternative title page and an appropriate
%%% binding margin,  [honors] for Honors College theses,
%%% and [durt] for undergraduate thesis submitted as part
%%% part of the distinction in mathematics program.
%%%%%%%%%%%%%%%%%%%%%%%%%%%%%%%%%%%%%%%%%%

%%%%%%%%%%%%%%%%%%%%%%%%%%%%%%%%%%%%%%%%%%
%%  LaTeX Preamble
%%%%%%%%%%%%%%%%%%%%%%%%%%%%%%%%%%%%%%%%%%
\begingroup\newif\ifmy
\IfFileExists{hche_h_plot_data.csv}{}{\mytrue}
\ifmy
\begin{filecontents*}{hche_h_plot_data.csv}
x,y,z
80,52,100
72,60,100
78,58,100
69,55,100
74,72,100
80,81,100
65,62,100
60,79,100
88,88,100
73,62,100
62,77,100
95,52,100
86,52,100
78,89,100
87,62,100
89,93,100
90,75,100
71,78,100
84,75,100
90,89,100
\end{filecontents*}
\fi\endgroup

\begingroup\newif\ifmy
\IfFileExists{lche_h_plot_data.csv}{}{\mytrue}
\ifmy
\begin{filecontents*}{lche_h_plot_data.csv}
x,y,z
60,2,100
52,10,100
78,8,100
60,5,100
74,22,100
80,31,100
65,12,100
60,29,100
88,38,100
73,12,100
62,27,100
95,12,100
86,2,100
78,39,100
87,12,100
59,23,100
80,25,100
81,28,100
84,25,100
90,29,100
\end{filecontents*}
\fi\endgroup

\begingroup\newif\ifmy
\IfFileExists{lcle_h_plot_data.csv}{}{\mytrue}
\ifmy
\begin{filecontents*}{lcle_h_plot_data.csv}
x,y,z
10,2,100
2,10,100
8,8,100
5,5,100
24,22,100
30,31,100
15,12,100
10,29,100
38,38,100
23,12,100
12,27,100
40,12,100
36,2,100
8,49,100
27,12,100
2,23,100
\end{filecontents*}
\fi\endgroup

\begingroup\newif\ifmy
\IfFileExists{hcle_h_plot_data.csv}{}{\mytrue}
\ifmy
\begin{filecontents*}{hcle_h_plot_data.csv}
x,y,z
10,52,100
2,60,100
8,58,100
5,55,100
24,72,100
30,81,100
15,62,100
10,79,100
18,88,100
28,90,100
23,62,100
12,77,100
37,62,100
48,66,100
36,52,100
8,99,100
27,62,100
2,73,100
\end{filecontents*}
\fi\endgroup

\begingroup\newif\ifmy
\IfFileExists{hche_l_plot_data.csv}{}{\mytrue}
\ifmy
\begin{filecontents*}{hche_l_plot_data.csv}
x,y,z
80,52,0
72,60,0
78,58,0
69,55,0
74,72,0
80,81,0
65,62,0
60,79,0
88,88,0
73,62,0
62,77,0
95,52,0
86,52,0
78,89,0
87,62,0
89,93,0
90,75,0
71,78,0
84,75,0
90,89,0
\end{filecontents*}
\fi\endgroup

\begingroup\newif\ifmy
\IfFileExists{lche_l_plot_data.csv}{}{\mytrue}
\ifmy
\begin{filecontents*}{lche_l_plot_data.csv}
x,y,z
60,2,0
52,10,0
78,8,0
60,5,0
74,22,0
80,31,0
65,12,0
60,29,0
88,38,0
73,12,0
62,27,0
95,12,0
86,2,0
78,39,0
87,12,0
59,23,0
80,25,0
81,28,0
84,25,0
90,29,0
\end{filecontents*}
\fi\endgroup

\begingroup\newif\ifmy
\IfFileExists{lcle_l_plot_data.csv}{}{\mytrue}
\ifmy
\begin{filecontents*}{lcle_l_plot_data.csv}
x,y,z
10,2,0
2,10,0
8,8,0
5,5,0
24,22,0
30,31,0
15,12,0
10,29,0
38,38,0
23,12,0
12,27,0
40,12,0
36,2,0
8,49,0
27,12,0
2,23,0
\end{filecontents*}
\fi\endgroup

\begingroup\newif\ifmy
\IfFileExists{hcle_l_plot_data.csv}{}{\mytrue}
\ifmy
\begin{filecontents*}{hcle_l_plot_data.csv}
x,y,z
10,52,0
2,60,0
8,58,0
5,55,0
24,72,0
30,81,0
15,62,0
10,79,0
18,88,0
28,90,0
23,62,0
12,77,0
37,62,0
48,66,0
36,52,0
8,99,0
27,62,0
2,73,0
\end{filecontents*}
\fi\endgroup
\begingroup\newif\ifmy
\IfFileExists{hcle_plot_data.csv}{}{\mytrue}
\ifmy
\begin{filecontents*}{hcle_plot_data.csv}
x,y
10,52
2,60
8,58
5,55
24,72
30,81
15,62
10,79
18,88
28,90
23,62
12,77
37,62
48,66
36,52
8,99
27,62
2,73
\end{filecontents*}
\fi\endgroup

\begingroup\newif\ifmy
\IfFileExists{lcle_plot_data.csv}{}{\mytrue}
\ifmy
\begin{filecontents*}{lcle_plot_data.csv}
x,y
10,2
2,10
8,8
5,5
24,22
30,31
15,12
10,29
38,38
23,12
12,27
40,12
36,2
8,49
27,12
2,23
\end{filecontents*}
\fi\endgroup

\begingroup\newif\ifmy
\IfFileExists{lche_plot_data.csv}{}{\mytrue}
\ifmy
\begin{filecontents*}{lche_plot_data.csv}
x,y
60,2
52,10
78,8
60,5
74,22
80,31
65,12
60,29
88,38
73,12
62,27
95,12
86,2
78,39
87,12
59,23
80,25
81,28
84,25
90,29
\end{filecontents*}
\fi\endgroup

\begingroup\newif\ifmy
\IfFileExists{hche_plot_data.csv}{}{\mytrue}
\ifmy
\begin{filecontents*}{hche_plot_data.csv}
x,y
80,52
72,60
78,58
69,55
74,72
80,81
65,62
60,79
88,88
73,62
62,77
95,52
86,52
78,89
87,62
89,93
90,75
71,78
84,75
90,89
\end{filecontents*}
\fi\endgroup
\usepackage[style=styles/uscauthoryear]{biblatex}
\usepackage{blindtext, graphicx, hyperref, color, caption}
\usepackage{tikz}
\usepackage{mathtools}
\usepackage{pgfplots}
\usepackage{xcolor}
\usepackage{amsmath}
\usepackage{amsfonts}
\usepackage{amssymb}
\usepackage{multicol}
\usepackage{algorithm}
\usepackage{algpseudocode}
\usepackage{amsthm}
\usepackage{amsmath}
\usepackage{fancyhdr}
\usepackage{lastpage}
\usepackage{hhline}

%%%%%%% FOR EDITING ONLY. REMOVE WHEN DONE %%%%%%%%%
%\openup 2em
%%%%%%%%%%%%%%%%%%%

\usetikzlibrary{calc,chains,arrows,shapes,arrows,backgrounds,positioning}
%
\usepackage[utf8]{inputenc}
\usepackage{array, booktabs}
\usepackage{colortbl}
\usepackage{csquotes}
\usepackage[acronym]{glossaries}
\pgfplotsset{compat=1.15}

\newcommand{\foo}{\makebox[0pt]{\textbullet}\hskip-0.5pt\vrule width 1pt\hspace{\labelsep}}
%
\newtheorem{definition}{Definition}
\newtheorem{example}{Example}[definition]

\newtheorem{corollary}{Corollary}
\newtheorem{lemma}{Lemma}
\newtheorem{proposition}{Proposition}
\newtheorem{theorem}{Theorem}

\renewcommand{\algorithmicrequire}{\textbf{Input:}}
\renewcommand{\algorithmicensure}{\textbf{Output:}}

\DeclarePairedDelimiter\abs{\lvert}{\rvert}

\tikzset{
block/.style={
  rectangle,
  draw,
  text width=4.5em,
  text centered,
  rounded corners,
  minimum height=3em},
line/.style={draw, -latex'},
edge/.style={draw},
webservice/.style={
  cloud,
  draw,
  cloud puffs=10,
  cloud puff arc = 120,
  aspect = 2,
  inner ysep =.5em}
}

\definecolor{bblue}{HTML}{4F81BD}
\definecolor{rred}{HTML}{C0504D}
\definecolor{ggreen}{HTML}{9BBB59}
\definecolor{ppurple}{HTML}{9F4C7C}
\definecolor{column_color}{HTML}{EFEFEF}

%%%%%%%%%%%%%%%%%%%%%%%%%%%%%%%%%%%%%%%%%%
%% You should include above
%% any LaTeX packages that you need.  Most packages should work 
%% with this documentclass.
%%%%%%%%%%%%%%%%%%%%%%%%%%%%%%%%%%%%%%%%%

\bibliography{references/references}


%%%%%%%%%%%%%%%%%%%%%%%%%%%%%%%%%%%%%%%%
%% The lines above specify a BibTeX style which controls 
%% the appearance of the bibliography and how citations to
%% the bibliography within the text will work.  It is based on the biblatex.sty
%% package and provides a Chicago style, as preferred by the Graduate School.
%% There are other acceptable styles.  Indeed, different academic disciplines
%% have different styles.
%% 
%% The line  \bibliography{references} will cause LaTeX is search for a file
%% called references.bib.  This file could be named differently.  For example
%% \bibliography{henry} would provoke a search for henry.bib.  The
%% file reference.bib (or henry.bib) is one you will have to produce.  It is
%% a BibTeX database of references you use.
%% 
%% There are a number of alternate ways to address your bibliographic needs.
%% See the documentation uscthesisdoc.pdf  for a discussion of the different options.
%%
%%
%% 
%%In any case, this  is a good spot to ask LaTeX to load what it needs to handle
%% literature citations and to layout the bibliography. 
%%
%%%%%%%%%%%%%%%%%%%%%%%%%%%%%%%%%%%%%%%%%%


\newtheorem{thm}{Theorem}[chapter]
\newtheorem*{thmun}{Theorem}
\newtheorem{cor}[thm]{Corollary}
\newtheorem{lem}[thm]{Lemma}
\theoremstyle{definition}
\newtheorem{defn}[thm]{Definition}
\newtheorem{ex}[thm]{Example}
\theoremstyle{plain}

\newcommand{\defCommitRate}[1]{
    \begin{definition}
    \label{#1}
     (Commit Rate) - the commit rate of a transaction $T_{i}$, denoted as  $C_{r}(T_{i})$, is calculated by the given formula:
     
     \[\textrm{$\abs{c_{i}}$} = \textrm{\# of executions of $T_{i}$ ending with a COMMIT result}\]
     \[\textrm{$\abs{a_{i}}$} = \textrm{\# of executions of $T_{i}$ ending with a ABORT result}\]
     \[\textrm{$C_{r}(T_{i})$} =\frac{\textrm{$\abs{c_{i}}$}}{\textrm{$\abs{c_{i} + a_{i}}$}}\]
     
     Once this ratio is calculated, it is compared to the overall ratio of commit to abort executions for all transactions.
     
    \end{definition}
}

\newcommand{\defEfficiencyRate}[1]{
    \begin{definition}
    \label{#1}
     (Efficiency Rate) - the efficiency rate of a transaction $T_{i}$, denoted as  $E_{r}(T_{i})$, is based on how its execution time compares to all transactions executed within the execution environment. The efficiency rate is calculated based on the given formula:
     
    \[\textrm{$E_{r}(T_{i})$} =\frac{\textrm{$AVG(AVG(T_{1}),...,AVG(T_{n}))$}}{\textrm{$AVG(T_{i})$}}\]
     
    \end{definition}
}
\newcommand{\createCategorizationGraph}[2]{
    % \begin{table}[h]
    % \captionsetup{justification=centering}
    % \centering
    % \begin{tabular}{l|c|c|}
    % \cline{2-3}
    %                                           & \multicolumn{1}{l|}{\textbf{Commit Rate ($C_{r}$)}} & \multicolumn{1}{l|}{\textbf{Efficiency Rate ($E_{r}$)}} \\ \hline
    % \multicolumn{1}{|l|}{\textbf{HCHE}}  & $>$ 50\%                          & $>$ 50\%                                \\ \hline
    % \multicolumn{1}{|l|}{\textbf{HCLE}}  & $>$ 50\%                       & $\le$ 50\%                               \\ \hline
    % \multicolumn{1}{|l|}{\textbf{LCHE}} & $\le$ 50\%                          & $>$ 50\%                                  \\ \hline
    % \multicolumn{1}{|l|}{\textbf{LCLE}} & $\le$ 50\%                       & $\le$ 50\%                                 \\ \hline
    % %\multicolumn{1}{|l|}{\textbf{No Trnd.}} &    \multicolumn{2}{c|}{See equation below}                                    \\ \hline
    % \end{tabular}
    % \caption{Transaction Categorization Bounds} % title of the Figure
    % \label{tbl:default_tmetrics} % label to refer figure in text
    % \end{table}
    
    \begin{figure}
    \captionsetup{justification=centering}
    \centering % used for centering Figure
    \begin{tikzpicture}
    \begin{axis}[
        title={Transaction Categorization Bounds},
        xlabel={Efficiency Rate (by percentage)},
        ylabel={Commit Rate (by percentage)},
        xmin=0, xmax=100,
        ymin=0, ymax=100,
        xtick={0,25,50,75,100},
        ytick={0,25,50,75,100},
        legend style={at={(0.03,0.5)},anchor=west},
        ymajorgrids=true,
        grid style=dashed,
    ]
    
    \draw[-] (0,50) -- (100,50);
    \draw[-] (50,0) -- (50,100);
    \addplot[
        only marks,
        color=red,
        mark=o,
        ] table [x=x, y=y, col sep=comma] {hcle_plot_data.csv};
    \addplot[
        only marks,
        color=red,
        mark=diamond,
        ] table [x=x, y=y, col sep=comma] {lcle_plot_data.csv};
    \addplot[
        only marks,
        color=red,
        mark=triangle,
        ] table [x=x, y=y, col sep=comma] {lche_plot_data.csv};
    \addplot[
        only marks,
        color=red,
        mark=square,
        ] table [x=x, y=y, col sep=comma] {hche_plot_data.csv};
    
    \legend{HCLE, LCLE, LCHE, HCHE}
     
    \end{axis}
    \end{tikzpicture}
    \caption{#2}
    \label{#1}
    \end{figure}
}

\newcommand{\createMLSCategorizationGraph}[2]{
    \begin{figure}
    \captionsetup{justification=centering}
    \centering % used for centering Figure
    \begin{tikzpicture}
    \begin{axis}[
        title={MLS Transaction Categorization Bounds},
        xlabel={Efficiency Rate},
        ylabel={Commit Rate},
        zlabel={Security (High/Low)},
        xmin=0, xmax=100,
        ymin=0, ymax=100,
        zmin=0, zmax=100,
        xtick={25,50,75,100},
        ytick={25,50,75,100},
        ztick={0,25,50,75,100},
        legend style={at={(1.39,0.68)},anchor=east},
        ymajorgrids=true,
        zmajorgrids=true,
        xmajorgrids=true,
        grid style=dashed,
    ]
    
    \draw[-] (0,50,50) -- (100,50,50);
    \draw[-] (50,0,50) -- (50,100,50);
    \draw[-] (0,0,50) -- (0,50,50);
    \draw[-] (0,0,50) -- (50,0,50);
    \draw[-] (0,100,50) -- (0,50,50);
    \draw[-] (0,100,50) -- (50,100,50);
    \draw[-] (100,0,50) -- (50,0,50);
    \draw[-] (100,0,50) -- (100,50,50);
    \draw[-] (100,100,50) -- (100,50,50);
    \draw[-] (100,100,50) -- (50,100,50);
    
    \addplot3+[
        only marks,
        color=blue,
        mark=o,
        ] table [x=x, y=y, z=z, col sep=comma] {hcle_h_plot_data.csv};
    \addplot3+[
        only marks,
        color=blue,
        mark=diamond,
        ] table [x=x, y=y, z=z, col sep=comma] {lcle_h_plot_data.csv};
    \addplot3+[
        only marks,
        color=blue,
        mark=triangle,
        ] table [x=x, y=y, z=z, col sep=comma] {lche_h_plot_data.csv};
    \addplot3+[
        only marks,
        color=blue,
        mark=square,
        ] table [x=x, y=y, z=z, col sep=comma] {hche_h_plot_data.csv};
    \addplot3+[
        only marks,
        color=red,
        mark=o,
        ] table [x=x, y=y, z=z, col sep=comma] {hcle_l_plot_data.csv};
    \addplot3+[
        only marks,
        color=red,
        mark=diamond,
        ] table [x=x, y=y, z=z, col sep=comma] {lcle_l_plot_data.csv};
    \addplot3+[
        only marks,
        color=red,
        mark=triangle,
        ] table [x=x, y=y, z=z, col sep=comma] {lche_l_plot_data.csv};
    \addplot3+[
        only marks,
        color=red,
        mark=square,
        ] table [x=x, y=y, z=z, col sep=comma] {hche_l_plot_data.csv};
    
    \legend{$HCLE_H$, $LCLE_H$, $LCHE_H$, $HCHE_H$, $HCLE_L$, $LCLE_L$, $LCHE_L$, $HCHE_L$}
     
    \end{axis}
    \end{tikzpicture}
    \caption{#2}
    \label{#1}
    \end{figure}
}

%%%%%%%%%%%%%%%%%%%%%%%%%%%%%%%%%%%%%%%%%%%%
%%  These are just a few sample lines. Put here any 
%%  commands of your own devising that you want to use.
%%  If these examples are no use to you, omit them.
%%%%%%%%%%%%%%%%%%%%%%%%%%%%%%%%%%%%%%%%%%%%%


%%%%%%%%%%%%%%%%%%%%%%%%%%%%%%%%%%%%%%%%%%%%%%%%%%%%%%
%%             The Front Matter
%%  The section below deals with the material that comes 
%%  before the actual content of the document: The title 
%%  page, abstract, acknowledgments,etc.
%%
%%  Some of it is required.
%%%%%%%%%%%%%%%%%%%%%%%%%%%%%%%%%%%%%%%%%%%%%%%%%%%%%%

\title{Correct Web Service Transactions in the Presence of Malicious and Misbehaving Transactions}

\author{John Thomas}{Ravan III}    %% First Name then 
                                 %% Last Name

\degreedate{2021}                      %% The year of graduation

%\month{December}                 %% Only for the honors option
                                 %% where it is REQUIRED

\otherdegrees{
Bachelor of Science\\
The Citadel, The Military College of South Carolina 2011\\ [\baselineskip]
Master of Science\\
The Citadel Graduate College 2014\\ %% The \\ on this line is 
}                                %% ESSENTIAL!

\degreename{Doctor of Philosophy}     %% The Graduate School provides 
                                 %% a list of official degrees.
\field{Computer Science}              %% Fields also provided by the 
                                 %% Graduate School.
\college{College of Engineering and Computing}  %%As listed by Grad School
\advisor[]{Dr.}{Csilla Farkas}{Advisor} %%% Be sure the 
\readera[]{Dr.}{Shankar M. Banik}{Co-Advisor}     %%% third field is 
\readerb[]{Dr.}{John R. Rose}{CSE}         %%% the one used in 
\readerc[]{Dr.}{Lannan Luo}{CSE} %%% your department.
\readerd[]{Dr.}{Jorge Crichigno}{IIT}               %%% Only use as many as
%%% If you have just two readers, for example, leave out \readerc and
%%% \readerd
%%%
%%% For Honors College theses use \reader{}{}   NO third field.
%%% The commands \otherdegrees, \degree, \field, \college, \readera, etc.
%%% are not used under the honors option.
%%%%%%%%%%%%%%%%%%%%%%%%%%%%%%%%%%%%%%%%%%%%%%%%%%%%%%%

\dean[]{Cheryl Addy}{Dean of The Graduate School}   %% The Dean of the Graduate School
                   %% BE SURE TO CHECK THE NAME OF THE
                   %%PERSON CURRENTLY HOLDING THIS POSITION	
		   %% and the correct title.		             		
                     %% For Honors College theses use
                     %% \schcsigner{}{}.  For example,
                     %% \schcsigner{Dr.}{Davis Baird}

\copyrightpage       %% This is optional. It makes a 
                     %% copyright page that will appear 
                     %% immediately after the title page.

\abstract{chapters/abstract}  %% This calls the file herkimer.tex but 
                     %% but you might replace herkimer by 
                     %% anything you like, for example by 
                     %% abstract. Note, the Graduate School
                     %% REQUIRES that PhD dissertations have 
                     %% abstracts.
                     %%
                     %% For Honors College theses use
                     %% \honorsabstract{}

%\summary{precis}     %% This command calls  precis.tex
                     %% It is only available with the honors
                     %% option and it is REQUIRED for Honors
                     %% theses. 

\acknowledgments{chapters/thanks} %% This calls the file thanks.tex 
%% This is optional       %% where you have put your 
                          %%acknowledgments.

\dedication{chapters/dedication}   %% Calls dedication.tex
%%% Also optional

\preface{chapters/forward}    %% Calls forward.tex.  Optional.

\makeLoT               %% Issue this command if your work has 
                       %% four or more tables.  A list of tables 
                       %% will be produced automatically.

\makeLoF               %% works the same way but for figures.

%%%%%%%%%%%%%%%%%%%%%%%%%%%%%%%%%%%%%%%%%%%%%%%%%%%%%%%%%%%%
%%  Finally, here is the meat.  The idea is to compose a 
%%  .tex file for each section of your thesis or dissertation.  
%%  Then use LaTeX's \include command to put them all together.  
%%  Doing it this way makes it easier to change the order of 
%%  exposition as your writing is in progress.  Also it
%%  makes it easy to print out just one section. The \include
%%  command always starts a new page. So every section would 
%%  start on a new page.  If you would like for sections just
%%  to continue, after the appropriate vertical space, on the
%%  current page, then use the \input command instead of the 
%%  \include command.
%%%%%%%%%%%%%%%%%%%%%%%%%%%%%%%%%%%%%%%%%%%%%%%%%%%%%%%%%%%%

%% Abbreviations
\glsaddall
\printglossary[title=List of Abbreviations, toctitle=List of Abbreviations]

\begin{document}
\printacronyms
\chapter{Introduction}\label{chap:intro}

Consistency among multiple interleaved transactions in a web service context has always been an issue for researchers and database administrators. Isolation and atomicity are two of the four \ac{ACID} properties that are often relaxed in order to prevent a performance bottleneck. However, when these properties are relaxed, the database can reach an inconsistent state when concurrent transactions interleave incorrectly. This causes data to become corrupted, expensive compensation transactions to be executed, and cascading rollbacks on multiple nodes to be completed before processing can continue. 

\section{Motivations}
By looking at a practical use case we can more clearly see the issue and the need for a solution that ensures consistency. In Figures \ref{fig:e_com_ticket} \& \ref{fig:bp_env}, we see five web services executing on three different database instances. The first four web services  create a common business process created by the \gls{bpel} \cite{BPEL}. The web services are: $WS_{1}$ (decrement inventory by product ID), $WS_{2}$ (process payment), $WS_{3}$ (add order by user ID), and $WS_{4}$ (delete user payment info). The goal of the process is to allow a customer to purchase a product from an e-Commerce site. $WS_{5}$ (delete user payment info) and $WS_{3}$ execute within the same database instance. With the relaxed properties in the web service context, concurrent executions of $WS_{5}$ and $WS_{3}$ could cause an inconsistent state on $Node_{3}$. This would then cause a cascading rollback to execute and revert the committed operations of $WS_{1}$ and $WS_{2}$. Existing research shows that many solutions have been presented in the past to address this issue (e.g., \cite{Fekete_SnapshotIso}, \cite{Alrifai_Distributed_Managment}, \cite{Fekete_RAMP}, \cite{Fekete_IsolationSupport}, \cite{Jacobi_Locking}, and \cite{Fekete_Promises}). 

The most influential research that inspired the prediction-based solution was the Promises Model. The Promises model presented by Alan Fekete et al. (e.g., \cite{Fekete_IsolationSupport} and \cite{Fekete_Promises}) is an elegant solution that "promises" a particular transaction that the requested resource will be available while allowing concurrent transactions to still execute on that resource. The Promises solution is robust in that it allows the "strengthening" or "weakening" of promises after they have already been made. This allows existing promises on resources to be modified without breaking the existing promise entirely. However, the solution introduces backwards compatibility issues along with a potential bottleneck at the occurrence of registering a promise for a particular transaction.

However, none of the existing work improves currency control based on the performance of the transactions. That is the likelihood that the transaction will commit and the computational cost of the transaction. In our work we provide improvements for concurrency control using these performance characteristics. 


\section{Contributions}
We provide a prediction-based solution to support efficient and consistent concurrency control. Our approach is based on building a reputation for each transaction using its efficiency rate (i.e., computational cost) and the outcome (i.e., commit or abort). Using these properties transactions are categorized into four categories. The priorities associated with each category impact the transaction's scheduling. Our aim is to prevent cascading rollbacks and inconsistent database state while supporting practical concurrency control. We provide new lock types corresponding to transaction categories. Using these locks transaction scheduler will be able to determine which lock requests to permit. These eventually determine transaction scheduling, delays, and aborts. Our expectation is that prediction-based scheduling will increase both efficiency and consistency.

We identified two research areas in the context of prediction-based scheduling within web service environments that need to be addressed. These are:
\begin{itemize}
    \item transactional correctness within concurrency control
    % \item predictions within multi-level secure databases
    \item dynamic reputation for transactions
    % \item prediction-based scheduler within linked databases
\end{itemize}

\subsection{Transactional Correctness}
In this work we developed the theoretical foundation for the prediction-based scheduling. This included the development of a framework, associated concepts, and technologies. A completely new concurrency control paradigm was developed in order to elevate particular transactions over others. In this paradigm there are three actions used to determine the course of action for a particular transaction. These three actions either grant, elevate, and decline an transaction to enable concurrent operations and prevent deadlock. By ensuring the transactional correctness within the prediction-based solution, we can then use this foundation to build upon in regards to other research areas. The work discussed in this area is documented in Chapter \ref{chap:prediction_based_scheduler}.

\begin{figure}[h]
\captionsetup{justification=centering}
\centering
\includegraphics[width=\textwidth]{images/SystemModel_Overall}
\caption{Overall System Model of Prediction-based Scheduler}
\label{fig:system_model_overall}
\end{figure}

\subsection{Dynamic Reputation for Transactions}
In both previous sections, the categorization of the transaction is assumed in order to continue forward with the decision model. This section of the dissertation involves the work needed to establish a dynamic reputation management system to allow for dynamic categorization. It involves building a reputation score for each transaction based on the transactions ranking of efficiency, commits, system aborts, and user aborts. Once a reputation score is provided we can then use dynamic reputation management to dynamically promote and demote transactions. This then allows the system to adapt to its environment dynamically. The work discussed in this area is documented in Chapter \ref{chap:dynamic_reputation}.

% Figure \ref{fig:system_model_overall} displays the system model for the work done for transactional correctness and building a dynamic reputation for transactions.
\section{Dissertation Outline}
The rest of this dissertation is organized as follows: Chapter \ref{chap:prediction_based_scheduler} outlines the research done in regards to transactional correctness while preserving concurrent operations within a web service environment. This work is published in \cite{ravan_ensuring_2020}. Chapter \ref{chap:dynamic_reputation} addresses the research done in order to build a reputation for a given transaction that is dynamic to its environment and its changing attributes. This work is submitted to ACM Transactions on Database Systems and awaiting response. Chapter \ref{chap:future_work} addresses the future work needed to adapt the prediction-based scheduler in Chapter \ref{chap:prediction_based_scheduler} and Chapter \ref{chap:dynamic_reputation} to a multi-level secure database and other possible solutions. Chapter \ref{chap:conclusion} contains the concluding remarks regarding all areas of research.
    %% Calls Introduction.tex
                          %% Honors theses are required to 
                          %% have an Introduction.  For
                          %% Honors theses, the file 
                          %% Introduction.tex should begin
                          %%
                          %% \chapter*{Introduction}
                          %% followed by the text of the 
                          %% introduction.
                          
\chapter{Prediction-Based Scheduler}\label{chap:prediction_based_scheduler}

\section{Overview}
\label{pbs:overview}
In this chapter, we present the foundation of the prediction-based scheduler. The problem defined here is the primary motivation for the current work and the anchor for the dissertation outline in Chapter \ref{chap:outline}. The work presented in Chapter \ref{chap:prediction_based_scheduler} is taken from \cite{ravan_ensuring_2020}.

\section{Introduction}
\label{diss:introduction}

The work needed for completion involves two problems that build upon the foundation of the existing work. In this chapter, we explain in detail the components needed for the completion of work. The first component is extending the prediction-based solution to multi-level secure databases by transitioning to a three-dimensional decision matrix. The second and final component is the work needed to dynamically manage the reputation of transactions executed within a prediction-based scheduler. We propose that a complete solution including both MLS databases and dynamic transaction reputation would satisfy all requirements needed for a Ph.D. All extensions listed in previous chapters will work towards prototypes for each problem in order to gather empirical results that prove the soundness of each solution.

\section{Problem Definition}
\label{sec:problem_definition}

In order to define the problem of stale transaction categories we must first discuss the problem of concurrent database transactions in a web service environment.

In traditional database systems, transactions are executed with \ac{ACID} properties to ensure correctness, durability, and consistency among all transactions that are executed on the system. When transactions are moved to a web service context where concurrent transactions occur frequently, the traditional model of transaction correctness is not feasible to deploy. Multiple interleaving transactions with the locking required in \ac{ACID} systems causes an overhead that is not acceptable for the end user. In order to accommodate concurrent transactions in a web service environment that execute in an acceptable time frame, locking is removed and transactions are allowed to execute and commit independently. While all transactions are executing and committing successfully then there are no solutions and the lack of locks works. However, in the event that a transaction fails and there are transactions that are dependent downstream then a cascading rollback occurs reverting the effects of the downstream transaction. All transactions are put on hold until a compensation transaction, generated by the scheduler to fix the results of the failed transaction, can execute successfully. This causes a lot of overhead in the system that can be avoided if the failed transaction can be isolated from dependent transactions.

In our previous work (see \cite{ravan_ensuring_2020}) we presented a prediction-based transaction scheduler that provides appropriate run times for web service environments. The scheduler predicts the outcome of a transaction based on the transactions execution history. The transaction is then placed into one of four categories based on whether commit rate and execution time. We then provided custom locking actions based on the transactional category. Transactions in categories with high commit rates and low execution times are allowed to execute concurrently while transactions in categories with low commit rates and long execution times are locked to prevent downstream effects.

Our previous work addresses the problem of cascading rollbacks and compensation transactions, however a new problem presents itself in a transaction that has been incorrectly categorized or its metrics have changed and it needs to be re-categorized. This becomes a problem when a transaction with a high commit rate is locked due to its category when it can execute concurrently without any undesired side effects. The other extreme and more disastrous use case is a transaction with a low commit rate that should be locked but executes concurrently with other transactions and causes those transactions to abort their executions. In these situations we need the ability to promote or demote a transaction as its execution history changes. 

As we discuss the use of the transaction's execution history, another problem presents itself; what do we do with transactions that are new to the system and do not have any execution history? If we are to address the problem of transactions with no execution history then a reputation score should take the place of the previous four category solution. An objective reputation score allows for a linear approach for transaction comparisons that provides two benefits; a more granular comparison (i.e. comparing transactions that would normally be within the same category and would previously conflict) and a default score for a transaction with no history. In the previous four category solution, there is no default category for transactions with no history. 

In our previous solution, we defined three different locking actions; grant (+), decline (-), and elevate ($\delta$). The decline action causes a transaction to wait for resources to become available. Due to the restrictive four categories of our previous solution, there are a large number of scenarios where a decline action would occur. By transitioning to a solution that is more granular, we could potentially turn decline actions into elevate actions. The elevate action will abort a transaction of a lower category in order for a transaction of a higher category to be granted access to needed resources. By design this prevents transactions that are well-behaving from being hindered by transactions that are not well-behaving. However, if we don't include the cost of an aborted transaction in our calculation then we could do more harm than good by elevating too frequently. Transitioning to a new metric will allow us too refine our rules for elevation to prevent elevate actions that cause more harm than benefit. See Tables \ref{tbl:read_lock_compatibility} and \ref{tbl:write_lock_compatibility} for reference. In the next section we walk through an airline ticketing use-case scenario to better explain the problems identified.

% This situation assumes that not only can the transaction be categorized but it can also be identified from other transactions executing in the system. Transaction identification will allow for a transaction to be identified from other transactions in the system and provide a uniqueness that can be pinpointed. This also shows that the transaction's execution should be translated into an objective reputation metric that can be scored as the execution history grows. 

\subsection{Use-Case Scenario}
\label{subsec:use_case_scenario}

In order to better explain the problems, let's look at a use case scenario of an airline ticketing system. Let's say we have two users that are attempting to reserve seats on an airline. The first user places a ticket, or a seat, in their shopping cart using an airline's online reservation system. While in the shopping cart, that seat is no longer available for reservation even though the transaction has not been completed. Simultaneously, a second user attempts to reserve a seat on the same airline and same flight. There are no seats available so that user is denied a reservation. Later on, the first user that placed the initial reservation in their shopping cart does not purchase the reservation in time. Their reservation is then expired and made available again.

In this scenario, the seat is made available due to the reconciliation efforts of the reservation system, however, in the end the airline loses profit. A seat that could have been purchased by the second user was not available due to the first user having placed the seat in their shopping cart. See the scenario diagram in Figure \ref{image:airline_reservation}. Figure \ref{image:airline_reservation_system_model} shows the same scenario with the transaction scheduler and database contained within the same logical unit. In both figures the solid lines represent the user transactions submitted to the database while the dotted lines represent the response from the transaction. The gray swim lanes labeled $T_0$, $T_1$, and $T_2$ show the transactions at certain time intervals.

\begin{figure}
\centering
\includegraphics[scale=0.50]{images/AirlineReservation.png}
\caption{Airline Reservation Use Case}
\label{image:airline_reservation}
\end{figure}

\begin{figure}
\centering
\includegraphics[scale=0.50]{images/AirlineReservation_Overview.png}
\caption{Airline Reservation Use Case System Model}
\label{image:airline_reservation_system_model}
\end{figure}

If the given scenario were to play out in a system that maintained the reputation of transactions entering the system, then the behavior of the first user would be tracked would be taken into consideration. The next time the user were to submit a similar transaction, that reputation would be taken into consideration and could potentially prevent the user from getting precedence over the seat reservation. This reinforces good user behavior in the system and also increases profit for the reservation system. In the next section, we discuss the related work that influenced the current problem and research.
\section{Related Work}
\label{rep:related_work}

There are many resources available for dynamic reputation management that shaped our understanding of the challenges of maintaining reputation of nodes/services. Many of the reputation systems were centered around maintaining a reputation in a distributed or decentralized system. These resources were \cite{clark_dynamic_2017}, \cite{de_paola_reputation_2008}, and  \cite{hu_reputation_2010}. While this helped our understand of the challenges of reputation management, our system would not experience the challenges associated with a decentralized trust management solution since the service and the database are both contained within the prediction-based scheduler itself. Other reputation management solutions were focused on the reputation or trustworthiness of web pages. Those solutions were presented in \cite{wang_research_2008} and \cite{melnikov_towards_2018}. These were also helpful but didn't address the reputation of database transactions directly which is the focus of the reputation of the prediction-based solution. The majority of reputation management systems are provided with the context of P2P or ad-hoc mobile networking in mind. Resources focused on dynamic reputations within networking environments are \cite{sun_dynamic_2019}, \cite{chiejina_dynamic_2014}, \cite{hu_reputation_2010}, and \cite{de_paola_reputation_2008}.

After studying the current environment of dynamic reputation systems, we believe this is a great opportunity to provide a dynamic reputation management solution particularly focused at database transactions within a web service environment.

%%%%%% BEGINNING OF SYSTEM MODEL SECTION %%%%%%%
\subsection{Definitions}
\label{sec:definitions}
This section outlines the definitions on which the solution is built. These definitions will be used to describe the different components that consist of the system model.

\begin{definition}
\label{def:compatibility}
(Compatibility) - a data item $d_{i}$ within transaction $T_{i}$ is locked in a non-compatible mode if:

\begin{enumerate}
  \item $d_{i}$ is locked by write-lock
  \item $Dominates_{S}(T_{i},T_{j})$ = false, or
  \item $Dominates_{W}(T_{i},T_{j})$ = false
\end{enumerate}
\end{definition}

\begin{definition}
\label{def:legal_scheduler}
 (Legal Scheduler) - a prediction-based scheduler is legal if:
 
 \begin{enumerate}
    \item \textbf{Grant:} (denoted as the addition symbol, +) a transaction is permitted to lock a data item if the data item is not already locked in a non-compatible mode (see Definition \ref{def:compatibility}) by an other transaction\footnote{In the event that a lock is requested of a resource that has not issued any locks, then the lock will be automatically granted. There is no conflict, and therefore, the compatibility matrices do not apply.}.
    \item \textbf{Decline:} (denoted as the subtraction symbol, -) a transaction $T_{i}$ is denied to lock a data item if the data item is already locked in a non-compatible mode (see Definition \ref{def:compatibility}) by another transaction $T_{j}$ where $\tau(T_{i}) \le \tau(T_{j})$
    \item \textbf{Elevate:} (denoted as lowercase delta, $\delta$) a transaction $T_{i}$ is permitted to lock a data item if all the non-compatible locks (see Definition \ref{def:compatibility}) on the data item are being held by transactions $T_{1}, ... , T_{k}$ such that $\tau(T_{j}) < \tau(T_{i})$ for all $j = 1, ..., k$ in this case $T_{1}, ... , T_{k}$ must first release the locks on the data item before $T_{i}$ is permitted to lock the data item.
 \end{enumerate}
\end{definition}

\begin{definition}
\label{def:commit_ranking}
(Commit Ranking) - the commit ranking of a transaction $T_{i}$, denoted as $CR_{T_{i}}$, is based on the number times a transaction commits compared to other transactions in the system. If $C_{T_{i}}$ is the number of commits for transaction $T_{i}$ and $T_{c}$ represents the subset of all transactions $T_{all}$ with smaller or equal number of commits than $T_{i}$ then the commit ranking is calculated based on the given formula: 
 
\[\textrm{$T_{c}$} = \{\, T\in \textrm{$T_{all}$} \mid \textrm{$C_{T} \leq C_{T_{i}}$} \,\}\]

\[\textrm{$CR_{T_{i}} = \frac{|T_{c}|}{|T_{all}|}$}\]

\begin{example}
Let $C_{T_{i}} = 500$, $|D| = 450$, and $|T_{all}| = 600$ then $CR_{T_{i}} = 0.75$
\end{example}
\end{definition}

\begin{definition}
\label{def:efficiency_ranking}
(Efficiency Ranking) - the efficiency ranking of a transaction $T_{i}$, denoted as $ER_{T_{i}}$, is based on how its execution time $time_{T_{i}}$ compares to all transactions executed within the execution environment. If $T_{e}$ represents the subset of all transactions $T_{all}$ with larger or equal execution time than $T_{i}$ then the efficiency ranking is calculated based on the given formula: 

\[\textrm{$T_{e}$} = \{\, T\in \textrm{$T_{all}$} \mid \textrm{$time_{T} \geq time_{T_{i}}$} \,\}\]

\[\textrm{$ER_{T_{i}} = \frac{|T_{e}|}{|T_{all}|}$}\]

\begin{example}
Let $time_{T_{i}} = 400s$, $|T_{e}| = 30$, and $|T_{all}| = 250$ then $ER_{T_{i}} = 0.12$
\end{example}
\end{definition}

\begin{definition}
\label{def:user_ranking}
(User Ranking) - the user ranking of a user in the system denoted as $UR_{i}$ is based on the user's ranking among other users and their impact on the commit rate (see Definition \ref{def:commit_ranking}) via forced aborts. If $FA_{U_{i}}$ is the amount of forced aborts caused by user $U_{i}$ and $U_{a}$ represents the subset of all users $U_{all}$ with more or equal number of forced aborts than $U_{i}$ then $UR_{i}$ is calculated on the given formula:

\[\textrm{$U_{a}$} = \{\, U\in \textrm{$U_{all}$} \mid \textrm{$FA_{U} \geq FA_{U_{i}}$} \,\}\]

\[\textrm{$FA_{U_{i}} = \frac{|U_{a}|}{|U_{all}|}$}\]

\begin{example}
Let $FA_{U_{i}} = 4$, $|U_{a}| = 15$, and $|U_{all}| = 24$ then $UR_{i} = 0.625$
\end{example}
\end{definition}

\begin{definition}
\label{def:system_abort_ranking}
(System Ranking) - the system ranking of a transaction $T_{i}$, denoted as $SR_{T_{i}}$, is the transaction's ranking among other transactions based on the number of times the transaction has been aborted by the system via elevate actions (see Definition \ref{def:legal_scheduler}) or any system causes. If $A_{T_{i}}$ represents the number of system aborts for transaction $T_{i}$ and $T_{a}$ represents the subset of all transactions $T_{all}$ with more or equal number of system aborts than $T_{i}$ then $SR_{T_{i}}$ is calculated on the given formula:

\[\textrm{$T_{a}$} = \{\, T\in \textrm{$T_{all}$} \mid \textrm{$A_{T} \geq A_{T_{i}}$} \,\}\]

\[\textrm{$SR_{T_{i}} = \frac{|T_{a}|}{|T_{all}|}$}\]

\begin{example}
Let $A_{T_{i}} = 7$, $|T_{a}| = 10$, and $|T_{all}| = 320$ then $SR_{T_{i}} = 0.03125$
\end{example}
\end{definition}

\begin{definition}
\label{def:reputation_score}
(Reputation Score) - a Reputation Score of a transaction $T_{i}$ denoted as $RS_{T_{i}}$ is a 4-tuple of the calculated attributes  for transaction $T_{i}$ and user $U_{T_{i}}$. The attributes are the commit ranking, efficiency ranking, user ranking, and the system ranking multiplied by a weighted value $w_{i}^{1...4}$ that change the precedence of a particular score value. The Reputation Score is denoted as follows:

\[\textrm{$RS_{T_{i}} = <w_{i}^{1}\times CR_{T_{i}},w_{i}^{2}\times ER_{T_{i}},w_{i}^{3}\times UR_{i},w_{i}^{4}\times SR_{T_{i}}>$}\]

where $CR_{T_{i}}$ is the Commit Ranking (see Definition \ref{def:commit_ranking}), $ER_{T_{i}}$ is the Efficiency Ranking (see Definition \ref{def:efficiency_ranking}), $UR_{i}$ is the User Ranking (see Definition \ref{def:user_ranking}), and $SR_{T_{i}}$ is the System Ranking (see Definition \ref{def:system_abort_ranking}).
\begin{example}
Assuming $w_{i}^{1...4} = 1$, for a transaction $T_{i}$ and a user $U_{i}$ let $CR_{T_{i}} = 0.35$, $ER_{T_{i}} = 0.75$, $UR_{i} = 0.23$, and $SR_{T_{i}} = 0.85$ then $RS_{T_{i}} = <0.35,0.75,0.23,0.85>$
\end{example}
\end{definition}

\begin{definition}
\label{def:strong_dominance}
(Strong Dominance) - Let $RS_{T_{i}} = <w_{i}^{1}\times CR_{T_{i}},w_{i}^{2}\times ER_{T_{i}},w_{i}^{3}\times UR_{i},w_{i}^{4}\times SR_{T_{i}}>$, and $RS_{T_{j}} = <w_{j}^{1}\times CR_{T_{j}},w_{j}^{2}\times ER_{T_{j}},w_{j}^{3}\times UR_{j},w_{j}^{4}\times SR_{T_{j}}>$ denote the reputation scores of transactions $T_{i}$ and $T_{j}$ respectively. Transaction $T_{i}$ has strong dominance over transaction $T_{j}$, denoted as $Dominates_{S}(T_{i},T{j})$, if and only if the following conditions are true: 

\begin{enumerate}
    \item $w_{i}^{1} \times CR_{T_{i}} \geq w_{j}^{1} \times CR_{T_{j}}$,
    \item $w_{i}^{2} \times ER_{T_{i}} \geq w_{j}^{2} \times ER_{T_{j}}$,
    \item $w_{i}^{3} \times UR_{i} \geq w_{j}^{3} \times UR_{j}$, and
    \item $w_{i}^{4} \times SR_{T_{i}} \geq w_{j}^{4} \times SR_{T_{j}}$
\end{enumerate}

\begin{example}
Assuming $w_{i}^{1...4} = 1$ and $w_{j}^{1...4} = 1$, for a transactions $T_{i}$ and $T_{j}$ and users $U_{i}$ and $U_{j}$ let $RS_{T_{i}} = <0.35,0.75,0.23,0.85>$ and $RS_{T_{j}} = <0.15,0.55,0.03,0.45>$ then $Dominates_{S}(T_{i},T{j})$
\end{example}
\end{definition}

\begin{definition}
\label{def:weak_dominance}
(Weak Dominance) - Let $SUM(T_{i}) = CR_{T_{i}} + ER_{T_{i}} + UR_{i} + SR_{T_{i}}$ and $SUM(T_{j}) = CR_{T_{j}} + ER_{T_{j}} + UR_{j} + SR_{T_{j}}$ denote the sum of the attributes of the reputation scores for transactions $T_{i}$ and $T_{j}$ respectively. Transaction $T_{i}$ has weak dominance over transaction $T_{j}$, denoted as $Dominates_{W}(T_{i},T{j})$, if and only if the following conditions are true:
\begin{enumerate}
    \item $Dominates_{S}(T_{i},T_{j})$ = false,
    \item $Dominates_{S}(T_{j},T_{i})$ = false, and
    \item $SUM(T_{i}) \geq SUM(T_{j})$
\end{enumerate}

\begin{example}
Assuming $w_{i}^{1...4} = 1$ and $w_{j}^{1...4} = 1$, for a transactions $T_{i}$ and $T_{j}$ and users $U_{i}$ and $U_{j}$ let $RS_{T_{i}} = <0.35,0.75,0.23,0.65>$ and $RS_{T_{j}} = <0.15,0.76,0.03,0.85>$ then $Dominates_{S}(T_{i},T{j}) = false$ but $SUM(T_{i}) = 1.98$ and $SUM(T_{j}) = 1.79$ therefore $Dominates_{W}(T_{i},T{j})$
\end{example}
\end{definition}

\begin{definition}
\label{def:not_comparable}
(Not Comparable) - Transaction $T_{i}$ is not comparable to transaction $T_{j}$ if and only if the following conditions are true:

\begin{enumerate}
    \item $Dominates_{S}(T_{i},T_{j})$ = false,
    \item $Dominates_{S}(T_{j},T_{i})$ = false,
    \item $Dominates_{W}(T_{i},T_{j})$ = false, and
    \item $Dominates_{W}(T_{j},T_{i})$ = false
\end{enumerate}

\begin{example}
Assuming $w_{i}^{1...4} = 1$ and $w_{j}^{1...4} = 1$, for a transactions $T_{i}$ and $T_{j}$ and users $U_{i}$ and $U_{j}$ let $RS_{T_{i}} = <0.35,0.75,0.23,0.65>$ and $RS_{T_{j}} = <0.25,0.76,0.12,0.85>$ then $Dominates_{S}(T_{i},T{j}) = false$. $SUM(T_{i}) = 1.98$ and $SUM(T_{j}) = 1.98$ and $Dominates_{W}(T_{i},T{j}) = false$ therefore $T_{i}$ has equal dominance over $T_{j}$
\end{example}
\end{definition}

% \begin{definition}
% \label{extrinsic_attributes}
%  (Extrinsic Attributes) - a transactional attribute is extrinsic if

%  \begin{enumerate}
%   \item it is mutable
%   \item it can be different among transactions of the same class, and
%   \item it affects the reputation of a transaction class
%  \end{enumerate}

%  \begin{example}
%  \label{ex_extrinsic_attribute}
%   Let $T_{a_{1}}$ be a transactional attribute of $T_{1}$ and $T_{2}$ that are within transactional class $T_{c_{1}}$. $T_{a_{1}}$ contains multiple values, changes, and determines the reputation of that transactional class. $T_{a_{1}}$ is an extrinsic attribute.
%  \end{example}
% \end{definition}

% \begin{definition}
% \label{transaction_class}
%  (Transaction Class) - transactions are considered to be in a transaction class if

%  \begin{enumerate}
%   \item they contain the same value for all intrinsic attributes, and
%   \item they do not exist in another transaction class
%  \end{enumerate}

%  \begin{example}
%  \label{ex_transaction_class}
%   Let $T_{1}$ and $T_{2}$ be two transactions in transaction class $T_{c_{1}}$. $T_{1}$ and $T_{2}$ have the same value for all intrinsic attributes and do not exist in another transaction class. $T_{c_{1}}$ is a properly identified transactional class.
%  \end{example}
% \end{definition}

\section{Environment}
\label{rep:environment}
\input{chapters/2__PBS_Scheduler/sections/system_model/transaction_metrics}
\input{chapters/2__PBS_Scheduler/sections/system_model/lock_compatibility}
\input{chapters/2__PBS_Scheduler/sections/system_model/rcds}
\subsection{Categorization Graph}
\label{pbs:cat_graph}

% The objective bounds that the system can be measured against must be set for commit and abort rate percentages and also for the upper and lower bounds for rate of efficiency. A static value could be used for these bounds but then the solution would not be flexible for different environments. Another considered solution could store the category bounds in a relation within the local database but then this approach places the responsibility on the database administrator to ensure the bounds are appropriate for the execution environment. In order to allow the categorization bounds to flex within the execution environment while also allowing the flexibility without the aid of an administrator, we created the categorization graph.

The categorization graph is a graphical representation of the transaction metrics. It is grouped into four sections where each section represents a category that a transaction can be placed. Categorization bounds (see Definition \ref{cat_bounds}) separate the graph into four sections and determine what category the transaction will receive once placed. Transactions are categorized based on the percentages of their previous execution metrics in comparison to the other transactions executing on the same system. This allows the system to flex accordingly with different environments. Depending on the metrics that each transaction possesses, the transactions will be categorized into the previous four categories listed above: $LCLE$, $HCLE$, $HCHE$, and $LCHE$. Transactions categorized as $LCLE$ must have an efficiency rate that is in the 50 percentile or lower where the commit rate is also in the 50 percentile or lower. Transactions categorized as $HCLE$ must have an efficiency rate that is also in the 50 percentile or lower, however, the commit rate must be in the 50 percentile or higher. $LCHE$ transactions must have an efficiency rate that is in the 50 percentile or higher where the commit rate is in the 50 percentile or lower. Transactions categorized as $HCHE$ must have an efficiency rate that is in the 50 percentile or higher where the commit rate is also in the 50 percentile or higher. Figure \ref{graph:cat_graph} and Definition \ref{transaction_categories} show the categorizations described above. The next section outlines the required algorithms for the prediction-based solution.

\createCategorizationGraph{graph:cat_graph}{Categorization Graph}

%%%%%% BEGINNING OF ALGORITHM SECTION %%%%%%%
\input{chapters/2__PBS_Scheduler/sections/algorithms/overview}
\input{chapters/2__PBS_Scheduler/sections/algorithms/determine_schedulers_action}
\input{chapters/2__PBS_Scheduler/sections/algorithms/execute_schedule}
\input{chapters/2__PBS_Scheduler/sections/algorithms/complexity_analysis}

%%%%%% BEGINNING OF ANALYSIS SECTION %%%%%%%
\input{chapters/2__PBS_Scheduler/sections/analysis/algorithm_analysis}
\input{chapters/2__PBS_Scheduler/sections/analysis/theoretical_contribution}

%%%%%% BEGINNING OF EXPERIMENTAL SIMULATION RESULTS SECTION %%%%%%%
\input{chapters/2__PBS_Scheduler/sections/experimental_simulation_results}

\section{Conclusion}
\label{sec:conclusion}

% In closing we conclude that this extension of the prediction-based scheduling solution involves the identification of two unique problems: identifying a class of transactions and maintaining a dynamic reputation for that class of transactions. Identifying a class of transactions allows PBS the ability to assign a categorization to transactions with common intrinsic attributes. Separating the intrinsic attributes from the extrinsic involves a flyweight modeling approach in order to properly model transaction classes. 

In closing we conclude that the dynamic reputation solution provides a system of greater reward and punishment than the previous prediction-based solution. The four category system of the prediction-based solution doesn't contain the level of granularity needed to establish dominance in conflicting situations. We also conclude that the overhead of the dynamic reputation solution is comparable to the previous prediction-based and \ac{2PL} solutions therefore establishing a resource management solution with a feasible overhead and consistent scheduling. By analyzing our findings from the experimentation we can conclude that execution environments with workloads of greater than 20\% conflicting transactions would benefit from the granularity of the dynamic reputation scheduler to provided consistency and efficient scheduling. %% Chapter 2
\chapter{Dynamic Transactional Reputation }
\label{chap:dynamic_reputation}

\section{Overview}
\label{drp:overview}
In this chapter, we present the dynamic reputation management system. The work contributed here extends the existing prediction-based scheduler presented in Chapter \ref{chap:prediction_based_scheduler}. The work here in Chapter \ref{chap:dynamic_reputation} is submitted to ACM Transactions on Information Systems and awaiting review.

\section{Introduction}
\label{sec:introduction}

Concurrency control is a problem that has always been at the forefront of researchers for some time now. In traditional database systems, the schedulers provide \ac{ACID} properties to transactions that are executed to ensure consistency and correctness. While this is the ideal solution for all transaction scheduling, it's not feasible when transactions are moved to a web service context. The overhead of the locking within web service transactions eliminates the use of traditional scheduling techniques. In order to increase efficiency of many concurrent transactions, the transactional properties of atomicity and isolation are relaxed to prevent the overhead of locking. While this increases efficiency of concurrent transactions, this places the database at a much higher risk of reaching an inconsistent state where data needs to be repaired. The current industry standard is to abandon locking and generate compensating transactions to fix the effects when consistency is lost however, compensating transactions can be very expensive when multiple conflicts occur. In previous work (\cite{ravan_ensuring_2020}) we developed a transaction scheduler that provided a prediction-based analytic to the transaction executing. We used transaction metrics from previous executions to place the transaction in a hierarchical category where we then provided targeted locking. For those transactions that were considered well-behaving, we allow concurrent executions with no locking but those that could potentially cause a cascading rollback, we provided locking to ensure that no other transactions were affected. Going forward in this work we'll simply refer to the prediction-based scheduler as PBS for brevity. 

In this work, we build upon the \ac{PBS} by focusing on the reputation management of the transactions. We use bit-wise scoring based on commit ranking (see Definition \ref{def:commit_ranking}),  efficiency ranking (see Definition \ref{def:efficiency_ranking}), user ranking (see Definition \ref{def:user_ranking}), and system ranking (see Definition \ref{def:system_abort_ranking}) of transactions entering the system. The bit-wise score is then used in a linear fashion to determine locking behaviors for those transactions. Higher scores receive precedence over lower scores therefore allowing for a more granular scheduler that was previously restricted to four categories (see \cite{ravan_ensuring_2020}).

% We focus on identifying classes of transactions (see Definition \ref{transaction_class}) based on the attributes of the transactions and then build a reputation score for those transaction classes. First, we separate the intrinsic (see Definition \ref{intrinsic_attributes}) and extrinsic (see Definition \ref{extrinsic_attributes}) attributes of transactions. The intrinsic attributes are used to identify a particular class of similar transactions. The extrinsic attributes are then used to create a score for the transaction class. 

The scores are calculated after each execution so that the most recent execution's metrics can be taken into consideration. This also prevents adding overhead to the transaction scheduler when transactions enter the system. The score will have already been calculated and a scheduling decision can be made at execution time. The following work details the reputation management system. This chapter is organized in the following order; Section \ref{sec:problem_definition} outlines the problem along with a use-case scenario. Section \ref{sec:related_work} discusses the existing research that has already taken place in regards to the problem. Section \ref{sec:system_model} outlines the system model for the solution. Section \ref{sec:empirical_results} illustrates the simulation results gathered from the prototype.


\section{Problem Definition}
\label{sec:problem_definition}

In order to define the problem of stale transaction categories we must first discuss the problem of concurrent database transactions in a web service environment.

In traditional database systems, transactions are executed with \ac{ACID} properties to ensure correctness, durability, and consistency among all transactions that are executed on the system. When transactions are moved to a web service context where concurrent transactions occur frequently, the traditional model of transaction correctness is not feasible to deploy. Multiple interleaving transactions with the locking required in \ac{ACID} systems causes an overhead that is not acceptable for the end user. In order to accommodate concurrent transactions in a web service environment that execute in an acceptable time frame, locking is removed and transactions are allowed to execute and commit independently. While all transactions are executing and committing successfully then there are no solutions and the lack of locks works. However, in the event that a transaction fails and there are transactions that are dependent downstream then a cascading rollback occurs reverting the effects of the downstream transaction. All transactions are put on hold until a compensation transaction, generated by the scheduler to fix the results of the failed transaction, can execute successfully. This causes a lot of overhead in the system that can be avoided if the failed transaction can be isolated from dependent transactions.

In our previous work (see \cite{ravan_ensuring_2020}) we presented a prediction-based transaction scheduler that provides appropriate run times for web service environments. The scheduler predicts the outcome of a transaction based on the transactions execution history. The transaction is then placed into one of four categories based on whether commit rate and execution time. We then provided custom locking actions based on the transactional category. Transactions in categories with high commit rates and low execution times are allowed to execute concurrently while transactions in categories with low commit rates and long execution times are locked to prevent downstream effects.

Our previous work addresses the problem of cascading rollbacks and compensation transactions, however a new problem presents itself in a transaction that has been incorrectly categorized or its metrics have changed and it needs to be re-categorized. This becomes a problem when a transaction with a high commit rate is locked due to its category when it can execute concurrently without any undesired side effects. The other extreme and more disastrous use case is a transaction with a low commit rate that should be locked but executes concurrently with other transactions and causes those transactions to abort their executions. In these situations we need the ability to promote or demote a transaction as its execution history changes. 

As we discuss the use of the transaction's execution history, another problem presents itself; what do we do with transactions that are new to the system and do not have any execution history? If we are to address the problem of transactions with no execution history then a reputation score should take the place of the previous four category solution. An objective reputation score allows for a linear approach for transaction comparisons that provides two benefits; a more granular comparison (i.e. comparing transactions that would normally be within the same category and would previously conflict) and a default score for a transaction with no history. In the previous four category solution, there is no default category for transactions with no history. 

In our previous solution, we defined three different locking actions; grant (+), decline (-), and elevate ($\delta$). The decline action causes a transaction to wait for resources to become available. Due to the restrictive four categories of our previous solution, there are a large number of scenarios where a decline action would occur. By transitioning to a solution that is more granular, we could potentially turn decline actions into elevate actions. The elevate action will abort a transaction of a lower category in order for a transaction of a higher category to be granted access to needed resources. By design this prevents transactions that are well-behaving from being hindered by transactions that are not well-behaving. However, if we don't include the cost of an aborted transaction in our calculation then we could do more harm than good by elevating too frequently. Transitioning to a new metric will allow us too refine our rules for elevation to prevent elevate actions that cause more harm than benefit. See Tables \ref{tbl:read_lock_compatibility} and \ref{tbl:write_lock_compatibility} for reference. In the next section we walk through an airline ticketing use-case scenario to better explain the problems identified.

% This situation assumes that not only can the transaction be categorized but it can also be identified from other transactions executing in the system. Transaction identification will allow for a transaction to be identified from other transactions in the system and provide a uniqueness that can be pinpointed. This also shows that the transaction's execution should be translated into an objective reputation metric that can be scored as the execution history grows. 

\subsection{Use-Case Scenario}
\label{subsec:use_case_scenario}

In order to better explain the problems, let's look at a use case scenario of an airline ticketing system. Let's say we have two users that are attempting to reserve seats on an airline. The first user places a ticket, or a seat, in their shopping cart using an airline's online reservation system. While in the shopping cart, that seat is no longer available for reservation even though the transaction has not been completed. Simultaneously, a second user attempts to reserve a seat on the same airline and same flight. There are no seats available so that user is denied a reservation. Later on, the first user that placed the initial reservation in their shopping cart does not purchase the reservation in time. Their reservation is then expired and made available again.

In this scenario, the seat is made available due to the reconciliation efforts of the reservation system, however, in the end the airline loses profit. A seat that could have been purchased by the second user was not available due to the first user having placed the seat in their shopping cart. See the scenario diagram in Figure \ref{image:airline_reservation}. Figure \ref{image:airline_reservation_system_model} shows the same scenario with the transaction scheduler and database contained within the same logical unit. In both figures the solid lines represent the user transactions submitted to the database while the dotted lines represent the response from the transaction. The gray swim lanes labeled $T_0$, $T_1$, and $T_2$ show the transactions at certain time intervals.

\begin{figure}
\centering
\includegraphics[scale=0.50]{images/AirlineReservation.png}
\caption{Airline Reservation Use Case}
\label{image:airline_reservation}
\end{figure}

\begin{figure}
\centering
\includegraphics[scale=0.50]{images/AirlineReservation_Overview.png}
\caption{Airline Reservation Use Case System Model}
\label{image:airline_reservation_system_model}
\end{figure}

If the given scenario were to play out in a system that maintained the reputation of transactions entering the system, then the behavior of the first user would be tracked would be taken into consideration. The next time the user were to submit a similar transaction, that reputation would be taken into consideration and could potentially prevent the user from getting precedence over the seat reservation. This reinforces good user behavior in the system and also increases profit for the reservation system. In the next section, we discuss the related work that influenced the current problem and research.
\section{Related Work}
\label{rep:related_work}

There are many resources available for dynamic reputation management that shaped our understanding of the challenges of maintaining reputation of nodes/services. Many of the reputation systems were centered around maintaining a reputation in a distributed or decentralized system. These resources were \cite{clark_dynamic_2017}, \cite{de_paola_reputation_2008}, and  \cite{hu_reputation_2010}. While this helped our understand of the challenges of reputation management, our system would not experience the challenges associated with a decentralized trust management solution since the service and the database are both contained within the prediction-based scheduler itself. Other reputation management solutions were focused on the reputation or trustworthiness of web pages. Those solutions were presented in \cite{wang_research_2008} and \cite{melnikov_towards_2018}. These were also helpful but didn't address the reputation of database transactions directly which is the focus of the reputation of the prediction-based solution. The majority of reputation management systems are provided with the context of P2P or ad-hoc mobile networking in mind. Resources focused on dynamic reputations within networking environments are \cite{sun_dynamic_2019}, \cite{chiejina_dynamic_2014}, \cite{hu_reputation_2010}, and \cite{de_paola_reputation_2008}.

After studying the current environment of dynamic reputation systems, we believe this is a great opportunity to provide a dynamic reputation management solution particularly focused at database transactions within a web service environment.
\section{System Model}
\label{rep:system_model}

This section of Chapter \ref{chap:dynamic_reputation} covers the system model in which the extension for VRM will be built. Here we build on the definitions, implementation, and environment that have been established in Sections \ref{pbs:system_model} \& \ref{mls:system_model} of the previous chapters.

\subsection{Definitions}
\label{sec:definitions}
This section outlines the definitions on which the solution is built. These definitions will be used to describe the different components that consist of the system model.

\begin{definition}
\label{def:compatibility}
(Compatibility) - a data item $d_{i}$ within transaction $T_{i}$ is locked in a non-compatible mode if:

\begin{enumerate}
  \item $d_{i}$ is locked by write-lock
  \item $Dominates_{S}(T_{i},T_{j})$ = false, or
  \item $Dominates_{W}(T_{i},T_{j})$ = false
\end{enumerate}
\end{definition}

\begin{definition}
\label{def:legal_scheduler}
 (Legal Scheduler) - a prediction-based scheduler is legal if:
 
 \begin{enumerate}
    \item \textbf{Grant:} (denoted as the addition symbol, +) a transaction is permitted to lock a data item if the data item is not already locked in a non-compatible mode (see Definition \ref{def:compatibility}) by an other transaction\footnote{In the event that a lock is requested of a resource that has not issued any locks, then the lock will be automatically granted. There is no conflict, and therefore, the compatibility matrices do not apply.}.
    \item \textbf{Decline:} (denoted as the subtraction symbol, -) a transaction $T_{i}$ is denied to lock a data item if the data item is already locked in a non-compatible mode (see Definition \ref{def:compatibility}) by another transaction $T_{j}$ where $\tau(T_{i}) \le \tau(T_{j})$
    \item \textbf{Elevate:} (denoted as lowercase delta, $\delta$) a transaction $T_{i}$ is permitted to lock a data item if all the non-compatible locks (see Definition \ref{def:compatibility}) on the data item are being held by transactions $T_{1}, ... , T_{k}$ such that $\tau(T_{j}) < \tau(T_{i})$ for all $j = 1, ..., k$ in this case $T_{1}, ... , T_{k}$ must first release the locks on the data item before $T_{i}$ is permitted to lock the data item.
 \end{enumerate}
\end{definition}

\begin{definition}
\label{def:commit_ranking}
(Commit Ranking) - the commit ranking of a transaction $T_{i}$, denoted as $CR_{T_{i}}$, is based on the number times a transaction commits compared to other transactions in the system. If $C_{T_{i}}$ is the number of commits for transaction $T_{i}$ and $T_{c}$ represents the subset of all transactions $T_{all}$ with smaller or equal number of commits than $T_{i}$ then the commit ranking is calculated based on the given formula: 
 
\[\textrm{$T_{c}$} = \{\, T\in \textrm{$T_{all}$} \mid \textrm{$C_{T} \leq C_{T_{i}}$} \,\}\]

\[\textrm{$CR_{T_{i}} = \frac{|T_{c}|}{|T_{all}|}$}\]

\begin{example}
Let $C_{T_{i}} = 500$, $|D| = 450$, and $|T_{all}| = 600$ then $CR_{T_{i}} = 0.75$
\end{example}
\end{definition}

\begin{definition}
\label{def:efficiency_ranking}
(Efficiency Ranking) - the efficiency ranking of a transaction $T_{i}$, denoted as $ER_{T_{i}}$, is based on how its execution time $time_{T_{i}}$ compares to all transactions executed within the execution environment. If $T_{e}$ represents the subset of all transactions $T_{all}$ with larger or equal execution time than $T_{i}$ then the efficiency ranking is calculated based on the given formula: 

\[\textrm{$T_{e}$} = \{\, T\in \textrm{$T_{all}$} \mid \textrm{$time_{T} \geq time_{T_{i}}$} \,\}\]

\[\textrm{$ER_{T_{i}} = \frac{|T_{e}|}{|T_{all}|}$}\]

\begin{example}
Let $time_{T_{i}} = 400s$, $|T_{e}| = 30$, and $|T_{all}| = 250$ then $ER_{T_{i}} = 0.12$
\end{example}
\end{definition}

\begin{definition}
\label{def:user_ranking}
(User Ranking) - the user ranking of a user in the system denoted as $UR_{i}$ is based on the user's ranking among other users and their impact on the commit rate (see Definition \ref{def:commit_ranking}) via forced aborts. If $FA_{U_{i}}$ is the amount of forced aborts caused by user $U_{i}$ and $U_{a}$ represents the subset of all users $U_{all}$ with more or equal number of forced aborts than $U_{i}$ then $UR_{i}$ is calculated on the given formula:

\[\textrm{$U_{a}$} = \{\, U\in \textrm{$U_{all}$} \mid \textrm{$FA_{U} \geq FA_{U_{i}}$} \,\}\]

\[\textrm{$FA_{U_{i}} = \frac{|U_{a}|}{|U_{all}|}$}\]

\begin{example}
Let $FA_{U_{i}} = 4$, $|U_{a}| = 15$, and $|U_{all}| = 24$ then $UR_{i} = 0.625$
\end{example}
\end{definition}

\begin{definition}
\label{def:system_abort_ranking}
(System Ranking) - the system ranking of a transaction $T_{i}$, denoted as $SR_{T_{i}}$, is the transaction's ranking among other transactions based on the number of times the transaction has been aborted by the system via elevate actions (see Definition \ref{def:legal_scheduler}) or any system causes. If $A_{T_{i}}$ represents the number of system aborts for transaction $T_{i}$ and $T_{a}$ represents the subset of all transactions $T_{all}$ with more or equal number of system aborts than $T_{i}$ then $SR_{T_{i}}$ is calculated on the given formula:

\[\textrm{$T_{a}$} = \{\, T\in \textrm{$T_{all}$} \mid \textrm{$A_{T} \geq A_{T_{i}}$} \,\}\]

\[\textrm{$SR_{T_{i}} = \frac{|T_{a}|}{|T_{all}|}$}\]

\begin{example}
Let $A_{T_{i}} = 7$, $|T_{a}| = 10$, and $|T_{all}| = 320$ then $SR_{T_{i}} = 0.03125$
\end{example}
\end{definition}

\begin{definition}
\label{def:reputation_score}
(Reputation Score) - a Reputation Score of a transaction $T_{i}$ denoted as $RS_{T_{i}}$ is a 4-tuple of the calculated attributes  for transaction $T_{i}$ and user $U_{T_{i}}$. The attributes are the commit ranking, efficiency ranking, user ranking, and the system ranking multiplied by a weighted value $w_{i}^{1...4}$ that change the precedence of a particular score value. The Reputation Score is denoted as follows:

\[\textrm{$RS_{T_{i}} = <w_{i}^{1}\times CR_{T_{i}},w_{i}^{2}\times ER_{T_{i}},w_{i}^{3}\times UR_{i},w_{i}^{4}\times SR_{T_{i}}>$}\]

where $CR_{T_{i}}$ is the Commit Ranking (see Definition \ref{def:commit_ranking}), $ER_{T_{i}}$ is the Efficiency Ranking (see Definition \ref{def:efficiency_ranking}), $UR_{i}$ is the User Ranking (see Definition \ref{def:user_ranking}), and $SR_{T_{i}}$ is the System Ranking (see Definition \ref{def:system_abort_ranking}).
\begin{example}
Assuming $w_{i}^{1...4} = 1$, for a transaction $T_{i}$ and a user $U_{i}$ let $CR_{T_{i}} = 0.35$, $ER_{T_{i}} = 0.75$, $UR_{i} = 0.23$, and $SR_{T_{i}} = 0.85$ then $RS_{T_{i}} = <0.35,0.75,0.23,0.85>$
\end{example}
\end{definition}

\begin{definition}
\label{def:strong_dominance}
(Strong Dominance) - Let $RS_{T_{i}} = <w_{i}^{1}\times CR_{T_{i}},w_{i}^{2}\times ER_{T_{i}},w_{i}^{3}\times UR_{i},w_{i}^{4}\times SR_{T_{i}}>$, and $RS_{T_{j}} = <w_{j}^{1}\times CR_{T_{j}},w_{j}^{2}\times ER_{T_{j}},w_{j}^{3}\times UR_{j},w_{j}^{4}\times SR_{T_{j}}>$ denote the reputation scores of transactions $T_{i}$ and $T_{j}$ respectively. Transaction $T_{i}$ has strong dominance over transaction $T_{j}$, denoted as $Dominates_{S}(T_{i},T{j})$, if and only if the following conditions are true: 

\begin{enumerate}
    \item $w_{i}^{1} \times CR_{T_{i}} \geq w_{j}^{1} \times CR_{T_{j}}$,
    \item $w_{i}^{2} \times ER_{T_{i}} \geq w_{j}^{2} \times ER_{T_{j}}$,
    \item $w_{i}^{3} \times UR_{i} \geq w_{j}^{3} \times UR_{j}$, and
    \item $w_{i}^{4} \times SR_{T_{i}} \geq w_{j}^{4} \times SR_{T_{j}}$
\end{enumerate}

\begin{example}
Assuming $w_{i}^{1...4} = 1$ and $w_{j}^{1...4} = 1$, for a transactions $T_{i}$ and $T_{j}$ and users $U_{i}$ and $U_{j}$ let $RS_{T_{i}} = <0.35,0.75,0.23,0.85>$ and $RS_{T_{j}} = <0.15,0.55,0.03,0.45>$ then $Dominates_{S}(T_{i},T{j})$
\end{example}
\end{definition}

\begin{definition}
\label{def:weak_dominance}
(Weak Dominance) - Let $SUM(T_{i}) = CR_{T_{i}} + ER_{T_{i}} + UR_{i} + SR_{T_{i}}$ and $SUM(T_{j}) = CR_{T_{j}} + ER_{T_{j}} + UR_{j} + SR_{T_{j}}$ denote the sum of the attributes of the reputation scores for transactions $T_{i}$ and $T_{j}$ respectively. Transaction $T_{i}$ has weak dominance over transaction $T_{j}$, denoted as $Dominates_{W}(T_{i},T{j})$, if and only if the following conditions are true:
\begin{enumerate}
    \item $Dominates_{S}(T_{i},T_{j})$ = false,
    \item $Dominates_{S}(T_{j},T_{i})$ = false, and
    \item $SUM(T_{i}) \geq SUM(T_{j})$
\end{enumerate}

\begin{example}
Assuming $w_{i}^{1...4} = 1$ and $w_{j}^{1...4} = 1$, for a transactions $T_{i}$ and $T_{j}$ and users $U_{i}$ and $U_{j}$ let $RS_{T_{i}} = <0.35,0.75,0.23,0.65>$ and $RS_{T_{j}} = <0.15,0.76,0.03,0.85>$ then $Dominates_{S}(T_{i},T{j}) = false$ but $SUM(T_{i}) = 1.98$ and $SUM(T_{j}) = 1.79$ therefore $Dominates_{W}(T_{i},T{j})$
\end{example}
\end{definition}

\begin{definition}
\label{def:not_comparable}
(Not Comparable) - Transaction $T_{i}$ is not comparable to transaction $T_{j}$ if and only if the following conditions are true:

\begin{enumerate}
    \item $Dominates_{S}(T_{i},T_{j})$ = false,
    \item $Dominates_{S}(T_{j},T_{i})$ = false,
    \item $Dominates_{W}(T_{i},T_{j})$ = false, and
    \item $Dominates_{W}(T_{j},T_{i})$ = false
\end{enumerate}

\begin{example}
Assuming $w_{i}^{1...4} = 1$ and $w_{j}^{1...4} = 1$, for a transactions $T_{i}$ and $T_{j}$ and users $U_{i}$ and $U_{j}$ let $RS_{T_{i}} = <0.35,0.75,0.23,0.65>$ and $RS_{T_{j}} = <0.25,0.76,0.12,0.85>$ then $Dominates_{S}(T_{i},T{j}) = false$. $SUM(T_{i}) = 1.98$ and $SUM(T_{j}) = 1.98$ and $Dominates_{W}(T_{i},T{j}) = false$ therefore $T_{i}$ has equal dominance over $T_{j}$
\end{example}
\end{definition}

% \begin{definition}
% \label{extrinsic_attributes}
%  (Extrinsic Attributes) - a transactional attribute is extrinsic if

%  \begin{enumerate}
%   \item it is mutable
%   \item it can be different among transactions of the same class, and
%   \item it affects the reputation of a transaction class
%  \end{enumerate}

%  \begin{example}
%  \label{ex_extrinsic_attribute}
%   Let $T_{a_{1}}$ be a transactional attribute of $T_{1}$ and $T_{2}$ that are within transactional class $T_{c_{1}}$. $T_{a_{1}}$ contains multiple values, changes, and determines the reputation of that transactional class. $T_{a_{1}}$ is an extrinsic attribute.
%  \end{example}
% \end{definition}

% \begin{definition}
% \label{transaction_class}
%  (Transaction Class) - transactions are considered to be in a transaction class if

%  \begin{enumerate}
%   \item they contain the same value for all intrinsic attributes, and
%   \item they do not exist in another transaction class
%  \end{enumerate}

%  \begin{example}
%  \label{ex_transaction_class}
%   Let $T_{1}$ and $T_{2}$ be two transactions in transaction class $T_{c_{1}}$. $T_{1}$ and $T_{2}$ have the same value for all intrinsic attributes and do not exist in another transaction class. $T_{c_{1}}$ is a properly identified transactional class.
%  \end{example}
% \end{definition}

% \section{Environment}
\label{rep:environment}
\section{Vector Reputation Management (VRM)}
\label{rep:vrm}

In Chapter \ref{chap:prediction_based_scheduler} we presented the transaction categorization graph (see Figure \ref{pbs:cat_graph}) that plots transactions on a graph based on their commit rate and efficiency rate. In Chapter \ref{chap:multi_level_security} we extended that graph by adding a third dimension of security classification to the graph (see Definition \ref{mls:transaction_categories}) causing a three-dimensional graphing of all transactional attributes (proposed in Section \ref{rep:remaining_work} in Figure \ref{graph:mls_cat_graph}). While the added dimension adapts the prediction-based solution to multi-level secure databases there is still the limitation that two transactions of the same categorization can be in conflict. In this situation, the prediction-based solution reverts to existing 2PL with no added benefit.

By transitioning to a solution that contains a continuous spectrum of categorization, we can avoid the situation where two transactions of the same categorization cause a conflict. This will then allow the existing rules of the prediction-based scheduler (see Tables \ref{tbl:read_lock_compatibility} \& \ref{tbl:write_lock_compatibility}) to execute and eliminate situations in which a decline action (see Definition \ref{legal_scheduler}) would normally be executed.
% \subsection{Algorithms}
\label{mls:algorithms}
% \section{Algorithms}
% \label{sec:algorithms}

% This section discusses all the algorithms needed for dynamic reputation management.
\section{Empirical Results}
\label{sec:empirical_results}

Here we discuss the setup and execution of the simulation for dynamic reputations. The prototype is used to generate simulation results and verify results of the executions.

\subsection{Application}
\label{sec:er_application}

In this section, we discuss the application environment used for executing transactions.

The application is written using Spring Boot and Java.
\href{https://spring.io/projects/spring-boot}{Spring Boot} is an application framework that allows Java applications to be containerized easily and manage dependencies within the application itself.

Within the application we have four transaction schedulers implemented

\begin{itemize}
  \item No-Locking Scheduler (NoSQL)
  \item Traditional Scheduler (\gls{2pl})
  \item Prediction-Based Scheduler'
  \item Dynamic Reputation Scheduler
\end{itemize}

Each scheduler is executed with the same execution parameters. Those parameters include

\begin{itemize}
    \item Users
    \item Transactions
    \item Rate of abort
    \item Rate of conflict
    \item Use Case (discussed in next section)
\end{itemize}

Each scheduler executes in its own dedicated thread and we record results for each execution for analysis. Table \ref{tbl:execution_history} is a schema diagram of the results that we capture for each transaction execution. Before we started running the schedulers we generated users and transactions with random rankings between 0 and 1 for an initial working set. We generated over 5800 users and over 11000 transactions.  When the recalculation percentage threshold is reached we recalculated all of the rankings based on our definitions of each ranking (see Section \ref{sec:definitions}). This recalculation takes place in its own thread to prevent blocking the schedulers from continuing with their executions.

\begin{table}
\caption{Execution History Attributes}
\captionsetup{justification=centering}
\centering
 \begin{tabular}{|p{0.28\linewidth} | p{0.5\linewidth}|}
 \hline
 \textbf{Attribute Name} & \textbf{Description} \\ [0.5ex] 
 \hline\hline
%  id & The primary key of the results table \\ 
%  \hline
 userid & Unique identifier for the user  \\
 \hline
 user\_ranking & Decimal value containing user ranking defined in Definition \ref{def:user_ranking}  \\
 \hline
 transactionid & Unique identifier for the transaction \\
 \hline
 commit\_ranking & Decimal value containing user ranking defined in Definition \ref{def:commit_ranking}  \\
 \hline
 system\_ranking & Decimal value containing user ranking defined in Definition \ref{def:system_abort_ranking}  \\
 \hline
 eff\_ranking & Decimal value containing user ranking defined in Definition \ref{def:efficiency_ranking}  \\
 \hline
 num\_of\_operations & Integer value of the number of operations in the transaction  \\
 \hline
 reputation\_score & String representation of all rankings together  \\
 \hline
 transaction\_exec\_time & Decimal representing to the total execution time in milliseconds  \\
 \hline
 percentage\_aborted & Decimal representing to percentage of aborted transactions over the total transactions during that particular execution  \\
 \hline
 recalculation\_needed & Boolean representing if the recalculation threshold was surpassed   \\
 \hline
 time\_executed & Timestamp representing the time of the execution  \\
 \hline
 dominance\_type & String representing what type of dominance was established (see Definitions \ref{def:strong_dominance}, \ref{def:weak_dominance}, and \ref{def:not_comparable})  \\
 \hline
 transaction\_outcome & String representing whether the execution committed, aborted, or aborted due to higher dominance  \\
 \hline
 scheduler\_type & String representing which scheduler submitted the execution  \\
 \hline
 use\_case & String representing what use case this execution was a part of  \\
 \hline
 category & String representing the category of the transaction if it was an execution from PBS  \\
 \hline
 transaction\_type & String representing whether it was as normal or compensation transaction  \\ [1ex] 
 \hline
\end{tabular}
\label{tbl:execution_history} % label to refer figure in text
\end{table}

% Once the application is deployed, it is running but the schedulers are not started. In order to start the application, we have to interact with the REST API that is a part of the application.

% \begin{itemize}
%   \item \textbf{/health}
%   \begin{itemize}
%      \item Returns a simple status if the application is running
%   \end{itemize}
%   \item \textbf{/info}
%   \begin{itemize}
%      \item Returns application version to ensure the right version of the application is running
%   \end{itemize}
%   \item \textbf{/start}
%   \begin{itemize}
%      \item This is the endpoint used to start the execution of the schedulers. There is a table that stores the use cases that we wish to run. Once this endpoint is accessed, it will pull the use case that we have configured and begin running with those parameters
%   \end{itemize}
%   \item \textbf{/stop}
%   \begin{itemize}
%      \item This stops the execution of the schedulers
%   \end{itemize}
%   \item \textbf{/update}
%   \begin{itemize}
%      \item This allows us to update the parameters without redeploying the application
%   \end{itemize}
% \end{itemize}




\subsection{Use Case Formulation}

We have experimented with a variety of use cases to fine tune our empirical results.  Table \ref{tbl:use_cases_initial} shows the use cases that were executed initially to get a baseline of how transactions within the system would be affected, when a recalculation would occur, and what was the impact of that recalculation. Table \ref{tbl:use_cases} shows our final use case selection for the performance measurement of our approach.
%Before the use cases executed in Table \ref{tbl:use_cases} were developed, there were thousands of transactions executed with a variety of parameters. These initial use cases helped us develop the ideal use cases that would best show the benefits and drawbacks of the new dynamic reputation solution.

For the graphs in this section we use the term \textbf{affected transactions} to identify transactions that were forced to abort by our scheduler.



\begin{table}
\caption{Initial Use Cases}
\captionsetup{justification=centering}
\centering
 \begin{tabular}{|| c | c | c | c | c ||} 
 \hline
 \textbf{Name} & \textbf{Total \#} & \textbf{Recalculation \%} &  \textbf{Conflict \%} & \textbf{Abort \%} \\ [0.5ex] 
 \hline\hline
 Use Case Alpha & $\approx$ 23,000 & 50 & 10 & 5  \\ 
 \hline
 Use Case Beta & $\approx$ 2,500 & 10 & 25 & 25  \\ 
 \hline
 Use Case Gamma & $\approx$ 1,900 & 5 & 40 & 40  \\ 
 \hline
 Use Case Delta & $\approx$ 1,200 & 7 & 50 & 50  \\ 
 [1ex] 
 \hline
\end{tabular}
\label{tbl:use_cases_initial} % label to refer figure in text
\end{table}

Our first preliminary use case (Use Case Alpha shown in Figure \ref{image:use_case_alpha}) started with a high recalculation percentage to get a baseline test of the prototype. The high recalculation percentage (50\%) caused that initially, the affected transactions spiked to approximately 3\%, and then began to level off as the number of transactions within the system increased. We observed that with the 50\% affected transactions rate, recalculation would never be triggered.

\begin{figure}
\centering
\includegraphics[scale=0.35]{images/UseCase1.png}
\caption{Use Case Alpha}
\label{image:use_case_alpha}
\end{figure}

Our next preliminary use case (Use Case Beta shown in Figure \ref{image:use_case_beta}) decreased the recalculation rate to 10\% while increasing the conflicting and abort percentages in the system to 25\%. This was enough of a change to cause a single recalculation in the beginning. As the number of transactions increased in the system the number of affected transactions would grow too slowly to initiate another recalculation.

\begin{figure}
\centering
\includegraphics[scale=0.35]{images/UseCase2.png}
\caption{Use Case Beta}
\label{image:use_case_beta}
\end{figure}

After seeing the results from Use Case Beta we set the parameters of the next execution to be a bit more exaggerated. Use Case Gamma (shown in Figure \ref{image:use_case_gamma}) lowers the recalculation percentage to 5\% while the conflicting and abort percentages were set much higher at 40\%. This caused, that the affected transactions quickly spiked to 100\%; which kicked off a recalculation of the rankings within the system. After the recalculation the numbers are reset, but the conflicting and abort percentages are so high that the execution continues spiking the affected transactions to 100\%. This repetition of spikes continues throughout the entire execution. While Use Case Gamma is not an ideal system that transactions will execute within, it gave us a upper bound of what happens when there is a large percentage of affected transactions.

\begin{figure}
\centering
\includegraphics[scale=0.35]{images/UseCase3.png}
\caption{Use Case Gamma}
\label{image:use_case_gamma}
\end{figure}

The final use case of our initial executions was Use Case Delta (shown in Figure \ref{image:use_case_delta}). This use case is very similar to the previous use case and the results show that as well. In this use case we increase the recalculation percentage to 7\% while also increasing the conflicting and abort percentages to 50\%. There is nothing significant to note here other than the findings met our expectations for the given the parameters.

\begin{figure}
\centering
\includegraphics[scale=0.35]{images/UseCase4.png}
\caption{Use Case Delta}
\label{image:use_case_delta}
\end{figure}

After seeing the results of the preliminary use cases we formulated the use cases documented in Table \ref{tbl:use_cases}.
%to better verify our expectations in Section \ref{sec:anal_expectations}. 
In the next sections we present the use cases and our empirical results.
\subsection{Use Cases}
\label{sec:er_use_cases}

In this section, we discuss the use cases that were used to simulate executions within the schedulers. Input parameters collected as a set make up a single use case.

Use cases are the parameters used for a specific execution. Use cases have the following attributes

\begin{itemize}
    \item \textbf{Name}
    \begin{itemize}
        \item A simple name to uniquely identify the use case
    \end{itemize}
    \item \textbf{Total Transactions Executed}
    \begin{itemize}
        \item An integer representing how many transactions were executed within that use case
    \end{itemize}
    \item \textbf{Recalculation Percentage}
    \begin{itemize}
        \item This is a decimal number representing the percentage of aborted transactions over all transactions that is the threshold of when a recalculation should occur throughout the system
    \end{itemize}
    \item \textbf{Conflicting Percentage}
    \begin{itemize}
        \item This is a decimal number representing the percentage of transactions that will conflict during execution.
    \end{itemize}
    \item \textbf{Abort Percentage}
    \begin{itemize}
        \item This is a decimal number representing the percentage of transactions that will end in an abort during execution.
    \end{itemize}
\end{itemize}


These act as the input parameters for the executions as each set of use case parameters cause a different load on the system. The varying load consists of how many times a recalculation occurs and how many times a transaction must be executed again due to conflict or an abort. Table \ref{tbl:use_cases} shows the use case parameters that were used for simulation.

\begin{table}
\captionsetup{justification=centering}
\centering
 \begin{tabular}{|| c | c | c | c | c ||} 
 \hline
 \textbf{Name} & \textbf{Total \#} & \textbf{Recalculation \%} &  \textbf{Conflict \%} & \textbf{Abort \%} \\ [0.5ex] 
 \hline\hline
 Use Case 1 & 5,000 & 30 & 0 & 10  \\ 
 \hline
 Use Case 2 & 5,000 & 30 & 50 & 10  \\ 
 \hline
 Use Case 3 & 5,000 & 30 & 25 & 10  \\ 
 \hline
 Use Case 4 & 5,000 & 30 & 20 & 10  \\ 
 \hline
 Use Case 5 & 5,000 & 30 & 22 & 10  \\ 
 \hline
 Use Case 6 & 5,000 & 30 & 75 & 10  \\ 
 \hline
 Use Case 7 & 5,000 & 30 & 100 & 10  \\ 
 [1ex] 
 \hline
\end{tabular}
\caption{Use Cases Executed}
\label{tbl:use_cases} % label to refer figure in text
\end{table}

Before the use cases executed in Table \ref{tbl:use_cases} there were numerous use cases executed as discovery metrics for the best results generation. The use cases listed in Table \ref{tbl:use_cases} outline the optimal executions to outline our contribution.

In the next section we discuss how the experimentation was executed, limitations of the experimentation, and the goal of the experimentation.
\subsection{Execution}
\label{sec:er_execution}

The goal of this section is to provide clarity around the execution of the experimentation and outline the goal of the experimentation in regards to our contribution.

The experimentation discussed in Sections \ref{sec:er_application} and \ref{sec:er_use_cases} outlines the application architecture and the use cases used as a part of the execution to submit the solution to different workloads. The execution itself (shown in Figure \ref{image:flow_of_execution}) involves the execution of all four schedulers simultaneously with the same workload.

\begin{figure}
\centering
\includegraphics[scale=0.30]{images/Scheduler_Flow.png}
\caption{Flow of Execution}
\label{image:flow_of_execution}
\end{figure}

The primary limitation of the execution involves the number of transactions and users executed each time. The schedulers are not designed as fully functional schedulers that can accept multiple transactions and users but are subsets of the schedulers that only accept two users and two transactions each time. The primary goal of the scheduler executions is to validate the algorithms of the dynamic reputation scheduler among other comparative schedulers. Writing subsets of the schedulers that only accept pairs of transactions and users was much more feasible to implement as a prototype rather than a fully functional system. The flow of execution for the dynamic reputation scheduler is outlined in Figure \ref{image:flow_of_drs} and the flow of the prediction based scheduler implemented in our previous work (see \cite{ravan_ensuring_2020}) is outlined in Figure \ref{image:flow_of_pbs}. The \ac{2PL} and NoSQL schedulers implement the standard algorithms that are defined.

\begin{figure}
\centering
\includegraphics[scale=0.28]{images/DRPScheduler.png}
\caption{Flow of Dynamic Reputation Scheduler}
\label{image:flow_of_drs}
\end{figure}

\begin{figure}
\centering
\includegraphics[scale=0.28]{images/PBSScheduler.png}
\caption{Flow of Prediction Based Scheduler}
\label{image:flow_of_pbs}
\end{figure}
\section{Analysis}
\label{sec:analysis}

When looking at the behavior of all the use case executions we did notice trends given the parameters that were used.

Use Cases 1 and 2 performed very similar with the parameters given. In both use cases, the number of aborted transactions never reached above 10\% of the total transactions therefore a recalculation of rankings was never executed. As more transactions executed within the system, the more the percentage of aborted transactions plateaued and we never saw a recalculation take place.

Use Cases 3 \& 4 were purposely executed with extreme parameters in order to see the load the system would endure with multiple recalculations. There were multiple recalculations executed in asynchronous threads. The executions did not block the execution of the transactions but caused a great deal of CPU usage. These are parameters that we would not expect in a live system but were used specifically to see the load of recalculations.

Use Cases 5 \& 6 were the use cases where all three schedulers were executing. All three schedulers executed with the same transactions, users, and use case parameters to get an equal comparison. This was the best indicator of what would be expected in a real world scenario.
\subsection{Expectations}
\label{sec:anal_expectations}

Before executing the simulation, our definitions (see Section \ref{sec:definitions}) and system model (see Section \ref{sec:system_model}) led us to certain expectations that would come from the simulation. The overall expectation and claim from the dynamic reputation solution is that it provides a low overhead resource management solution with consistent scheduling. That claim is supported by the following three expectations:

\begin{enumerate}
    \item There will be greater reward and punishment from the dynamic reputation solution than from our previous prediction-based solution
    \item We will see the reward and punishment reflected in the rankings of the users and transactions
    \item The execution time will be comparable to both the previous prediction-based and \ac{2PL} scheduler to not indicate a serious overhead
\end{enumerate}
\subsection{Findings}
\label{sec:anal_findings}

Our first finding defends our first claim that there will be greater reward and punishment in our new solution than our previous prediction-based solution. We can determine a greater level of reward and punishment by looking at the percentage of transactions that were elevated due to a conflict. Our formula for calculating the percentage of elevated transactions is below:

\[\textrm{$T_{elevate}$} = \textrm{\# of executions that caused an ELEVATE}\]
\[\textrm{$T_{total}$} = \textrm{Total \# of executions}\]
\[\textrm{$P_{reward}$} =\frac{\textrm{$T_{elevate}$}}{\textrm{$T_{total}$}} \times 100\]
 
After analyzing the results we discovered that the $P_{reward}$ for the dynamic reputation solution is 51.9\% and the $P_{reward}$ for the prediction-based solution is 7.1\%. This is expected given that the reputation management solution involves a much more granular approach to establishing dominance that the four category system in the previous prediction-based solution. Therefore, this confirms that the dynamic reputation solution allows for a greater percentage of reward and punishment among the conflicting transactions.

Our second finding defends our second claim that we will see the reward and punishment reflected in the rankings of the users and transactions. By seeing a variance in the rankings of the users and transactions (see Definitions \ref{def:commit_ranking}, \ref{def:efficiency_ranking}, \ref{def:user_ranking},  and \ref{def:system_abort_ranking}) then we can confirm that our recalculations are causing changes in dominance based on the reputations of users and transactions. Figure \ref{image:variance_of_transaction_rankings} and \ref{image:variance_of_user_rankings} are two graphs showing the variance in the growth/reduction of transaction and user rankings across executions. You can see the rankings shrinking and growing as recalculations occur. Figure \ref{image:variance_of_transaction_rankings} shows the transaction rankings changing throughout the system. The y-axis represents the variance. The x-axis is the transaction ID (not shown for view ability). This graph represents all transactions that executed more than once within the system in order to show the changes in rankings between executions.

Figure \ref{image:variance_of_user_rankings} shows the same variance but for user rankings as they change throughout the system due to recalculation. The graph shows that the reward and punishment is actively being applied throughout the system.

\begin{figure}
\centering
\includegraphics[scale=0.20]{images/VarianceofTransactionRankings.png}
\caption{Variance of Transaction Rankings}
\label{image:variance_of_transaction_rankings}
\end{figure}

\begin{figure}
\centering
\includegraphics[scale=0.20]{images/VarianceofUserRankings.png}
\caption{Variance of User Rankings}
\label{image:variance_of_user_rankings}
\end{figure}

Our third and final finding defends our third claim that the execution time will be comparable to both the previous prediction-based and \ac{2PL} schedulers. When comparing all of the schedulers execution time over different workloads (see Figure \ref{image:scheduler_comparison}) we can see the differing execution times however, even with the differing execution times the overhead of the dynamic reputation system is comparable and a feasible solution. This defends our third and final claim that the dynamic reputation solution is a feasible solution but as we examine the data we can be more precise of when the solution should be legitimately considered.

After examining the data we see that the execution time is directly related to the workload. The defining difference of the workloads is the percentage of conflicting transactions in the workload. As the percentage of conflict increases in the differing workloads you can see the schedulers begin to execute at different execution times. 

From the graph in Figure \ref{image:scheduler_comparison} we can deduce that the best environment for the dynamic reputation solution is within execution environments that contain greater than 20\% conflicting transactions. Execution environments with high levels of conflict can benefit from the dynamic reputation solution while environments with less than 20\% conflict would be impacted by the overhead of processing within the dynamic reputation solution.

\begin{figure}
\centering
\includegraphics[scale=0.60]{images/SchedulerComparison.png}
\caption{Scheduler Comparison}
\label{image:scheduler_comparison}
\end{figure}

With our findings defending all three of our claims we can with confidence claim that the dynamic reputation solution provides a low overhead resource management solution with consistent scheduling.

\section{Conclusion}
\label{sec:conclusion}

% In closing we conclude that this extension of the prediction-based scheduling solution involves the identification of two unique problems: identifying a class of transactions and maintaining a dynamic reputation for that class of transactions. Identifying a class of transactions allows PBS the ability to assign a categorization to transactions with common intrinsic attributes. Separating the intrinsic attributes from the extrinsic involves a flyweight modeling approach in order to properly model transaction classes. 

In closing we conclude that the dynamic reputation solution provides a system of greater reward and punishment than the previous prediction-based solution. The four category system of the prediction-based solution doesn't contain the level of granularity needed to establish dominance in conflicting situations. We also conclude that the overhead of the dynamic reputation solution is comparable to the previous prediction-based and \ac{2PL} solutions therefore establishing a resource management solution with a feasible overhead and consistent scheduling. By analyzing our findings from the experimentation we can conclude that execution environments with workloads of greater than 20\% conflicting transactions would benefit from the granularity of the dynamic reputation scheduler to provided consistency and efficient scheduling. %% Chapter 3
\chapter{Dissertation Outline}
\label{chap:outline}

% \section{Introduction}
\label{diss:introduction}

The work needed for completion involves two problems that build upon the foundation of the existing work. In this chapter, we explain in detail the components needed for the completion of work. The first component is extending the prediction-based solution to multi-level secure databases by transitioning to a three-dimensional decision matrix. The second and final component is the work needed to dynamically manage the reputation of transactions executed within a prediction-based scheduler. We propose that a complete solution including both MLS databases and dynamic transaction reputation would satisfy all requirements needed for a Ph.D. All extensions listed in previous chapters will work towards prototypes for each problem in order to gather empirical results that prove the soundness of each solution.

% \input{chapters/5__Dissertation_Outline/sections/mls_databases}
% \input{chapters/5__Dissertation_Outline/sections/dynamic_reputation}
% \section{Prediction-Based Scheduling within Linked Databases}
\label{diss:linked_databases}

In this section we discuss the proposed work for implementing the comprehensive prediction-based scheduler (Chapters \ref{chap:prediction_based_scheduler} \& \ref{chap:multi_level_security}) into a linked-database environment. This section contains the proposal itself. The background information, problem definition, and existing research for the problem is outlined in Chapter \ref{chap:pbs_w_link_db}.

The final problem to attempt is the adaptation of the Prediction-based solution within a linked database environment. The challenges of scaling the solution to a linked database environment still needs to researched and the problems defined so that a proper solution can be presented. We believe there is a problem to be formally defined and analyzed through the lens of the Prediction-based solution.

Our proposal for linked database systems is much like the proposal outline in Section \ref{proposal:mls_database_scheduling}. Again, we propose researching the current academic environment for what is common practice among linked database systems, providing a solution based on the outcome of that research, verifying the accuracy, and working toward a viable prototype with initial results. Once again, since the bulk of the theoretical and simulation results have been accomplished in Chapter \ref{chap:prediction_based_scheduler} through the publication of \cite{ravan_ensuring_2020} we propose that this work consist of a submission to a conference.
\section{Dissertation Outline}
\label{diss:dissertation_outline}

\newcommand{\dissPhaseOne}
{\textbf{Transactional Correctness:} Consistency control algorithms prototyped and formally proven as a part of prediction-based solution (\gls{pbs}). \textbf{Published in IEEE Transactions, February 2020}\newline}

\newcommand{\dissPhaseTwo}
{\textbf{Dissertation Proposal}}

\newcommand{\dissPhaseThree}
{\textbf{Dynamic Transaction Reputation:} Analyze and leverage reputation management solutions within \gls{pbs} \textbf{Submitted to ACM Transactions of Information Systems\newline}}

\newcommand{\dissPhaseFour}
{\textbf{Dissertation Defense}}

In this section we explain the full dissertation outline of milestones that were accomplished and their timing.

\subsection{Outline}

Table \ref{table:dissertation_outline} shows the dissertation in all of its phases. Research began in August of 2014 after becoming enrolled as a student. I met with Dr. Farkas initially and began working on the transactional-correctness portion of the work that transitioned into the prediction-based scheduler. This is the work detailed in Chapter \ref{chap:prediction_based_scheduler}. The work underwent multiple revisions until we were confident with the solution and the form it has now taken. The work was completed and submitted to IEEE Transactions on Services Computing in April of 2019. In January of 2020, we received a decision of Accepted with Minor Revisions. Revisions were made and that has now been published. This serves as the bulk of the work and also the foundation of the extensions proposed in Chapters \ref{chap:future_work}. Starting in February of 2020, we begin work on the dynamic transaction reputations. That work was completed in August of 2021 and submitted to ACM Transactions on Information Systems. We are still awaiting response.

\begin{table}
\centering
\renewcommand\arraystretch{1.4}
\captionsetup{singlelinecheck=false, labelfont=sc, labelsep=quad, justification=centering}
\caption{Outline}\vskip -1.5ex
\begin{tabular}{@{\,}r <{\hskip 2pt} !{\foo} >{\raggedright\arraybackslash}p{10cm}}
\toprule
\addlinespace[1.5ex]
February 2020 & \dissPhaseOne\\
August 2020 & \dissPhaseTwo\\
September 2020 & \dissPhaseThree\\
September 2021 & \dissPhaseFour\\
\end{tabular}
\label{table:dissertation_outline}
\end{table}
\section{Conclusion}
\label{sec:conclusion}

% In closing we conclude that this extension of the prediction-based scheduling solution involves the identification of two unique problems: identifying a class of transactions and maintaining a dynamic reputation for that class of transactions. Identifying a class of transactions allows PBS the ability to assign a categorization to transactions with common intrinsic attributes. Separating the intrinsic attributes from the extrinsic involves a flyweight modeling approach in order to properly model transaction classes. 

In closing we conclude that the dynamic reputation solution provides a system of greater reward and punishment than the previous prediction-based solution. The four category system of the prediction-based solution doesn't contain the level of granularity needed to establish dominance in conflicting situations. We also conclude that the overhead of the dynamic reputation solution is comparable to the previous prediction-based and \ac{2PL} solutions therefore establishing a resource management solution with a feasible overhead and consistent scheduling. By analyzing our findings from the experimentation we can conclude that execution environments with workloads of greater than 20\% conflicting transactions would benefit from the granularity of the dynamic reputation scheduler to provided consistency and efficient scheduling. %% Chapter 4
\chapter{Future Work}\label{chap:future_work}

\section{Introduction}
\label{mls:introduction}

In this Chapter we discuss the potential for future work and extensions of the prediction-based scheduler. Primarily we discuss the work involved with multi-level secure databases.

After the framework for transactional correctness has been developed (work published in \cite{ravan_ensuring_2020}) within the prediction-based solution we can then begin extending the two-dimensional categorization of the prediction-based scheduler to a categorization for multi-level secure database. The added dimension would include the security classification and allow for a much more robust decision-model. This portion of the overall solution would focus on multi-level secure database systems and covert timing channels. By providing a third dimension to the existing framework we can therefore extend our categorization structure. This allows us to provide a cover story for the timing difference of transactions with differing security classifications. 

The problem with multi-level secure databases is the possibility that there could be a covert channel allowing unauthorized access. The covert channel would provided the ability for a transaction of a lower security classification to access resources designed for a higher security classification. Existing research provides possible solutions but many of the solutions starve transactions of higher security classifications from gaining access to the resources needed (see Section \ref{mls:related_work}). With that cover story in mind we can then provide a solution that elevates high security transactions would the presence of a covert timing channel. With the prediction-based solution provided in Chapter \ref{chap:prediction_based_scheduler}, we can provide a cover story that determines locking priority based on the reputation of the transaction as a whole. This prevents the starvation of higher security classification transactions. By taking into consideration the security classification as a metric to calculate the reputation of the transaction, we also prevent the presence of covert channels. 
\section{Problem Definition}
\label{sec:problem_definition}

In order to define the problem of stale transaction categories we must first discuss the problem of concurrent database transactions in a web service environment.

In traditional database systems, transactions are executed with \ac{ACID} properties to ensure correctness, durability, and consistency among all transactions that are executed on the system. When transactions are moved to a web service context where concurrent transactions occur frequently, the traditional model of transaction correctness is not feasible to deploy. Multiple interleaving transactions with the locking required in \ac{ACID} systems causes an overhead that is not acceptable for the end user. In order to accommodate concurrent transactions in a web service environment that execute in an acceptable time frame, locking is removed and transactions are allowed to execute and commit independently. While all transactions are executing and committing successfully then there are no solutions and the lack of locks works. However, in the event that a transaction fails and there are transactions that are dependent downstream then a cascading rollback occurs reverting the effects of the downstream transaction. All transactions are put on hold until a compensation transaction, generated by the scheduler to fix the results of the failed transaction, can execute successfully. This causes a lot of overhead in the system that can be avoided if the failed transaction can be isolated from dependent transactions.

In our previous work (see \cite{ravan_ensuring_2020}) we presented a prediction-based transaction scheduler that provides appropriate run times for web service environments. The scheduler predicts the outcome of a transaction based on the transactions execution history. The transaction is then placed into one of four categories based on whether commit rate and execution time. We then provided custom locking actions based on the transactional category. Transactions in categories with high commit rates and low execution times are allowed to execute concurrently while transactions in categories with low commit rates and long execution times are locked to prevent downstream effects.

Our previous work addresses the problem of cascading rollbacks and compensation transactions, however a new problem presents itself in a transaction that has been incorrectly categorized or its metrics have changed and it needs to be re-categorized. This becomes a problem when a transaction with a high commit rate is locked due to its category when it can execute concurrently without any undesired side effects. The other extreme and more disastrous use case is a transaction with a low commit rate that should be locked but executes concurrently with other transactions and causes those transactions to abort their executions. In these situations we need the ability to promote or demote a transaction as its execution history changes. 

As we discuss the use of the transaction's execution history, another problem presents itself; what do we do with transactions that are new to the system and do not have any execution history? If we are to address the problem of transactions with no execution history then a reputation score should take the place of the previous four category solution. An objective reputation score allows for a linear approach for transaction comparisons that provides two benefits; a more granular comparison (i.e. comparing transactions that would normally be within the same category and would previously conflict) and a default score for a transaction with no history. In the previous four category solution, there is no default category for transactions with no history. 

In our previous solution, we defined three different locking actions; grant (+), decline (-), and elevate ($\delta$). The decline action causes a transaction to wait for resources to become available. Due to the restrictive four categories of our previous solution, there are a large number of scenarios where a decline action would occur. By transitioning to a solution that is more granular, we could potentially turn decline actions into elevate actions. The elevate action will abort a transaction of a lower category in order for a transaction of a higher category to be granted access to needed resources. By design this prevents transactions that are well-behaving from being hindered by transactions that are not well-behaving. However, if we don't include the cost of an aborted transaction in our calculation then we could do more harm than good by elevating too frequently. Transitioning to a new metric will allow us too refine our rules for elevation to prevent elevate actions that cause more harm than benefit. See Tables \ref{tbl:read_lock_compatibility} and \ref{tbl:write_lock_compatibility} for reference. In the next section we walk through an airline ticketing use-case scenario to better explain the problems identified.

% This situation assumes that not only can the transaction be categorized but it can also be identified from other transactions executing in the system. Transaction identification will allow for a transaction to be identified from other transactions in the system and provide a uniqueness that can be pinpointed. This also shows that the transaction's execution should be translated into an objective reputation metric that can be scored as the execution history grows. 

\subsection{Use-Case Scenario}
\label{subsec:use_case_scenario}

In order to better explain the problems, let's look at a use case scenario of an airline ticketing system. Let's say we have two users that are attempting to reserve seats on an airline. The first user places a ticket, or a seat, in their shopping cart using an airline's online reservation system. While in the shopping cart, that seat is no longer available for reservation even though the transaction has not been completed. Simultaneously, a second user attempts to reserve a seat on the same airline and same flight. There are no seats available so that user is denied a reservation. Later on, the first user that placed the initial reservation in their shopping cart does not purchase the reservation in time. Their reservation is then expired and made available again.

In this scenario, the seat is made available due to the reconciliation efforts of the reservation system, however, in the end the airline loses profit. A seat that could have been purchased by the second user was not available due to the first user having placed the seat in their shopping cart. See the scenario diagram in Figure \ref{image:airline_reservation}. Figure \ref{image:airline_reservation_system_model} shows the same scenario with the transaction scheduler and database contained within the same logical unit. In both figures the solid lines represent the user transactions submitted to the database while the dotted lines represent the response from the transaction. The gray swim lanes labeled $T_0$, $T_1$, and $T_2$ show the transactions at certain time intervals.

\begin{figure}
\centering
\includegraphics[scale=0.50]{images/AirlineReservation.png}
\caption{Airline Reservation Use Case}
\label{image:airline_reservation}
\end{figure}

\begin{figure}
\centering
\includegraphics[scale=0.50]{images/AirlineReservation_Overview.png}
\caption{Airline Reservation Use Case System Model}
\label{image:airline_reservation_system_model}
\end{figure}

If the given scenario were to play out in a system that maintained the reputation of transactions entering the system, then the behavior of the first user would be tracked would be taken into consideration. The next time the user were to submit a similar transaction, that reputation would be taken into consideration and could potentially prevent the user from getting precedence over the seat reservation. This reinforces good user behavior in the system and also increases profit for the reservation system. In the next section, we discuss the related work that influenced the current problem and research.
\section{Related Work}
\label{mls:related_work}

Multi-level secure databases are a huge area of research due to the security concerns that can arise within these databases and the consequences if a vulnerability is exposed. The consequences have been so great that many users of multiple security classifications use multiple databases with duplicated common resources to prevent any security vulnerability from happening \footnote{This implementation is commonly found in DoD database systems and their Security Technical Implementation Guides (STIGs) \hyperlink{https://public.cyber.mil/stigs/}{https://public.cyber.mil/stigs/}}. However, researchers understand the benefits of having a secure solution contained in a single database system with multiple security levels.

Jajodia et. al. is one of the main motivations for our work (see \cite{jajodia_fair_1998}). In this publication we see a new locking protocol presented specifically for multi-level secure databases that prevents the starvation of lower security classification transactions. In this work, the researchers analyzed the existing two-phase locking protocol with additional policy additions to increase performance. In their analysis they discovered in order to increase performance and prevent starvation by increasing the fairness of all transactions, they needed to restrict the number of low security transactions executing. One quote from the work that motivates our work is,

\begin{displayquote}
"Several concurrency control algorithms that are free from covert channels have been proposed in the literature. Most of these algorithms prevent covert timing channels by ensuring that transactions at lower security levels are never delayed by the actions of a transaction at a higher security level. This can be accomplished by providing a higher priority to low transactions whenever a data conflict occurs between a high transaction and a low transaction."
\end{displayquote}

The prediction-based solution established in Chapter \ref{chap:prediction_based_scheduler} provides a solution to elevate or demote transactions based on transactional attributes that are deemed necessary for the transaction's reputation score. As a part of the reputation score, the security level in which a transaction resides can be a part of the transaction's attributes necessary for ranking.

Other, more recent, works that have been of influence for this solution involve \cite{mahmoud_encryption_2019} by Mahmoud and Alqumboz, \cite{sun_access_2011} by Ying-Guan Sun, \cite{hedayati_evaluation_2010} by Hedayati et. al., \cite{shanwal_secure_2013} by Shanwal and Kumar, \cite{sapra_development_2014} by Sapra et. al., \cite{kaur_performance_2004} by Kaur, N. et. al., \cite{costich_analysis_1991} by Costich, O.L. et. al., \cite{kaur_feedback_2007} by Kaur, N. et. al, \cite{keefe_database_1993} by Keefe T.F. et. al, and \cite{david_secure_1993} by David N. et. al. All of which have been built up on the work of the Bell–LaPadula Model, Biba Integrity Model, and lattice based security model (LBAC) (work referenced in \cite{bell_secure_1973}, \cite{biba_integrity_1977}, \& \cite{denning_lattice_1976}). An overview of multilevel secure databases and transaction processing can be found in Atluri et. al. (\cite{atluri_mls}).
\section{Environment}
\label{mls:environment}

A multi-level secure database is much like any traditional database system. The major difference is the presence of resources, users, and transactions with differing security levels. This can be resources within the system that contain a certain a security level. This can also be true for users who maintain a certain security level and therefore the transactions generated by the user contains a certain security level. Current architecture solutions leverage the Bell-LaPadula model (\cite{bell_secure_1973}) to ensure that current security levels are maintained and data is not accessed inappropriately. The Bell LaPadula Model abides by two main rules to ensure secure data access. The two rules are:

\begin{enumerate}
  \item A subject at a given security level may not read an object at a higher security level. This is known as the Simple Security Property
  \item A subject at a given security level many not write to any object a lower security level. This is known as the * (star) Property
\end{enumerate}

\begin{figure}
\centering
\includegraphics[scale=0.45]{images/BellLapadulaModel.png}
\caption{Bell-LaPadula Model}
\label{fig:bell_lapadula_model}
\end{figure}

Both of these properties are shown in Figure \ref{fig:bell_lapadula_model}. While this ensures proper data access control for security levels, the Bell-LaPadula Model doesn't protect against covert channels that occur due to concurrent transactions. Concurrent transactions cause issues when there are conflicting operations. When there is a conflict, one of the transactions must wait for the other transactions to finish processing before execution can continue processing. The presence (or even absence) of a time delay for the transaction to execute introduces a covert channel. 

A covert channel is a security flaw where a means of communication to transmit unauthorized information is available via the normal means of communication. Timing channels are a form of covert channel where the presence of absence of a execution delay conveys unauthorized information about the underlying system. This work is documented by Girling in \cite{girling_covert_1987}. Timing channels are difficult to prevent and many times requires review of the application source code directly to ensure all operations execute with the same timing delay. Common solutions to prevent covert timing channels in multi-level secure databases when there are conflicting operations is to abort the transaction with a higher security classification. This prevents the transaction with a lower security transaction from detecting a timing delay. The timing delay would communicate to the lower security transaction that resources of a higher security classification were present and therefore leaking unauthorized information. Figure \ref{fig:env_covert_channel_exposure} (referenced from Figure \ref{fig:ws_trans_starvation} in Section \ref{mls:problem_definition}) shows the exposure of a timing covert channel.

\begin{figure}
\centering
\includegraphics[scale=0.45]{images/TransactionStarvation.jpg}
\caption{Transaction Starvation Exposes Timing Covert Channel}
\label{fig:env_covert_channel_exposure}
\end{figure}
\section{Transaction Quality Measure}
\label{mls:tqm}

In this section we discuss the potential future work for implementing a security classification within the prediction-based scheduler (outlined in Chapters \ref{chap:prediction_based_scheduler} and \ref{chap:dynamic_reputation}) that would address the issues found specifically in \gls{mls} databases. 

Within the prediction-based solution, there are currently four attributes that allow for a reputation score to be calculated among transactions. Those four attributes are commit ranking, efficiency ranking, user ranking and system ranking (see Definitions \ref{def:commit_ranking}, \ref{def:efficiency_ranking}, \ref{def:user_ranking}, and \ref{def:system_abort_ranking} in Chapter \ref{chap:dynamic_reputation}). This allows for the formation of a reputation score to be calculated for each transaction in the system. Within these reputation scores, there is a dominance structure that causes transactions to be prioritized depending on if dominance can be established (outlined in Definitions \ref{def:strong_dominance} and \ref{def:weak_dominance}). With all of these components in place, we have a foundation for efficient transaction categorization within \gls{mls} databases.

A potential future state of the system is to use the existing reputation score within the existing prediction-based solution alongside a security label to create a two-tuple Transaction Quality Measure (TQM). This will involve an update dominance structure to ensure the absence of covert channels and also prevent starvation of higher security transactions within the system. The solution would allow for a better decision making model for which transactions are aborted and rescheduled. The new Transaction Quality Measure would take security classification into account, but it would not allow the security classification to be the only dictating factor. Extending the prediction-based solution would allow the other four attributes to be included within the decision process. Figure \ref{fig:existing_reputation_score} is a representation of the existing reputation score defined in Chapter \ref{chap:dynamic_reputation}. Figure \ref{fig:mls_strong_dominance} shows the strong dominance structure of Transaction Quality Measures where $SL$ is the security label and $PV$ is the performance vector representing the existing reputation score. Figure \ref{fig:mls_weak_dominance} shows the weak dominance structure.

\begin{figure}[h]
\captionsetup{justification=centering}
\centering % used for centering Figure

\[\textrm{$RS_{T_{i}}$ = $<w_{i}^{1}\times CR_{T_{i}},w_{i}^{2}\times ER_{T_{i}},w_{i}^{3}\times UR_{i},w_{i}^{4}\times SR_{T_{i}}>$}\]

\caption{Current Reputation Score} % title of the Figure
\label{fig:existing_reputation_score} % label to refer figure in text

\end{figure}

\begin{figure}[h]
\captionsetup{justification=centering}
\centering % used for centering Figure

\[\textrm{
$TQM_{1}(SL_{1},PV_{1}) \geq TQM_{2}(SL_{2},PV_{2})$ iff $SL_{1} \leq SL_{2}$ and $PV_{1} \geq PV_{2}$}\]

\caption{MLS Strong Dominance} % title of the Figure
\label{fig:mls_strong_dominance} % label to refer figure in text

\end{figure}

\begin{figure}[h]
\captionsetup{justification=centering}
\centering % used for centering Figure

\[\textrm{
$TQM_{1}(SL_{1},PV_{1}) \geq TQM_{2}(SL_{2},PV_{2})$ iff $SL_{1} \geq SL_{2}$ and $PV_{1} > PV_{2}$}\]

\caption{MLS Weak Dominance} % title of the Figure
\label{fig:mls_weak_dominance} % label to refer figure in text

\end{figure}

In order to prevent covert timing channels within multi-level secure database, the future solution would leverage the timing delay of the existing prediction-based solution to be used as a "cover story" for the timing difference between transactions of differing security levels. The cover story would allow for transactions to be aborted for conflicting transactions without introducing a covert timing channel for unauthorized disclosure of high security resources.

In summary, there are two well-known problems within multi-level secure databases. The first problem is the existence timing covert channels when transactions of multiple security levels are accessing a common resource. The presence or absence of a time delay provides the indication of high security resources that are available. The second problem, is brought on by the solution to multi-level secure databases. A solution to prevent against covert channels is to abort transactions of a higher security classification so that the time delay does not exist. However, this then causes these transactions to suffer from starvation and will never be executed. The solution presented in this section will address both problems and provide a way forward for more granular decision-making within multi-level secure database systems.

With the prediction-based scheduler in place and a solution for dynamic reputation of transactions, the possibilities for extension within multi-level secure databases is then feasible . Chapter \ref{chap:prediction_based_scheduler} presented the solution and operated under the assumption that the reputation of the transactions were already established. Chapter \ref{chap:dynamic_reputation} focuses on exactly how the transactions establish their reputation and also dynamically increase or decrease their reputation. With this work in place, extending the prediction-based system to multi-level secure databases is now possible.
\section{Additional Future Work}

In this section, we want to outline the future work opportunities of the prediction-based scheduler and how the work can be expanded upon. The work mentioned in this section is not meant to be included in this dissertation, but rather listing outstanding opportunities for the current and proposed work to continue forward.

\subsection{Snapshot Isolation}
Another potential for future work is the ability to perform snapshot isolation within the different categorizations of transactions. This extension will be for both malicious and lower priority transactions that affect the majority of well-performing transactions. In this work we'll use snapshot isolation to execute certain categorizations of transactions on snapshots of the database in order to prevent the effects of low-performing transactions from affecting all transactions. Once the outcome of a transaction has been determined then the snapshot can either be discarded or merged.

\subsection{Prediction-based Scheduling within Linked Databases}
An additional extension is the issue of efficient concurrency control within linked database environments. Currently the Prediction-based solution addresses efficient concurrent transactions within a web-service environment that are contained within a single cluster. This particular area of research will address the problem through the lens of the Prediction-based solution. The difficulty of the problem within this work is adapting the existing framework of correctness built within the Prediction-based solution so that it will scale to linked database systems while preserving its existing capabilities. Figure \ref{fig:system_model_linked_databases} shows the system model for the Prediction-based solution within linked database systems.

\begin{figure}[h]
\captionsetup{justification=centering}
\centering
\includegraphics[width=\textwidth]{images/LinkedDatabase_SystemModel}
\caption{Prediction-based Scheduler within Linked Databases}
\label{fig:system_model_linked_databases}
\end{figure}

\subsection{PostgreSQL \& MySQL}
\label{conclusion:posgressql}
Two very commonly used databases within enterprise applications are \href{https://www.postgresql.org/}{PostgresSQL} and \href{https://www.mysql.com/}{MySQL}. Both of which are open-source relational databases where their code is available to the public for modification and contribution. Open-source applications tend to have very difficult review process which allows for quality code and reliable software. Both of these database management systems have provided their code on Github so that the community can contribute features, bug fixes, and enhancements accordingly. The code for PostgresSQL is located at \href{https://github.com/postgres/postgres}{https://github.com/postgres/postgres} and the code for MySQL is located at \href{https://github.com/mysql/mysql-server}{https://github.com/mysql/mysql-server}.

One opportunity for future work would be to fork one or both of these repositories on a controlled system and implement the algorithms of the prediction-based scheduler within the database management system itself. Currently, the prediction-based scheduler has been proven theoretically in a test environment using an in-memory database. This work has proven the viability of the solution and the consistency that it provides. By placing the prediction-based solution in the database management system itself, it would provide a beautiful marriage of academia and industry coming together for a common solution. The initial goal would be to get the algorithms working on a mirrored fork initially in a controlled test environment, then moving that solution to a clustered environment to ensure scalability, and eventually providing an official pull request of the prediction-based scheduler to the code maintainers of both systems so the solution would then be available to the general public in future releases. This future work provides a direct road map from academic theory to impacting the global industry for the benefit of the masses.
\section{Conclusion}
\label{mls:conclusion}
In summary, there are a multitude of opportunities to extend the current work into new realms. The prediction-based scheduler along with dynamic transaction reputation provides a system of consistency and scalability that can be extended into multi-level secure databases, linked databases, and other systems of databases. This contribution will provide a foundation for future researchers to extend into realms of database scheduling and transaction execution that have not been discussed before.
 %% Chapter 5

\chapter{Conclusion}
\label{chap:conclusion}

\section{Concluding Remarks}
\label{conclusion:concluding_remarks}
In this proposal, we have first analyzed the shortcomings to existing web service database transactions. These shortcomings involve the need for compensation transactions that, ultimately, are unnecessary overhead if A.C.I.D transaction properties can be leveraged. We analyzed the current solutions involving compensation transactions and maintaining consistency within a web service environment. As a result of this analysis, research and prototyping led to the development of a prediction-based solution. The final solution dynamically uses different concurrency control mechanisms depending on the attributes of the transaction. In the current work we have formally proven that this solution will ensure consistency by using dynamic concurrency control mechanisms (published in \cite{ravan_ensuring_2020}).

Second, we have proposed the needed extensions to the work that will address three additional areas of improvement. The first area is expanding the prediction-based solution to a multi-level secure database environment where multiple security classifications exist. This involves extending the existing two-dimensional categorization graph to a three-dimensional graph for all transactional attributes to prevent the existence of covert timing channels that can appear due to differing security classifications. By doing so we provide a cover story for the transactional environment that enables the promotion of high security transactions without the disclosure of a covert timing channel.

The final effort will be the final extension leveraging all phases of the dissertation. This effort will focus on the formal categorization of transactions before entering the prediction-based solution. In this effort we extract the extrinsic attributes from the intrinsic attributes to reveal transactional classes that can be identified and grouped. From there, those classes contain a reputation that is managed dynamically in order to grow with the environment. That then allows the categorization of transactions to be promoted and demoted based on the reputation.

% The final effort will be the final extension leveraging all solutions. This effort will use Snapshot Isolation for low-performing and malicious transactions. With the current and proposed research the overall contribution will ensure consistency among web service transactions. The contribution will also minimize the effects of malicious and poorly performing transactions within a web service environment.
% \input{chapters/6__Conclusion/sections/future_work}     %% Honors theses are required to 
                          %% have an unnumbered chapter
                          %% for conclusions.  The file
                          %% Conclusion.tex should begin
                          %%   
                          %% \chapter*{Conclusion}
                          %% followed by the appropriate
                          %% text.

\printbibliography %%  This is the command to use to
			       %%  insert the bibliography if you are using
                           %% the biblatex.sty package.  See the 
                           %% uscthesisdoc.pdf documentation for
                           %% for alternative bibliographic systems.     

\Appendix                 %% Use this command if you have one 
                          %% appendix. Use \Appendices if you 
                          %% have more than one.
	
\input{chapters/appendix}         %% Calls toolong.tex which contains
                          %% an appendix. After issuing the 
                        %% command \Appendix or \Appendices
                        %% you must use \input not \include
                        %% to load the first appendix.
\end{document}
%%%%%%%%%%%%%%%%%%%%%%%%%%%%%%%%%%%%%%%%%%%%%%%%%%%%%%%%%%%%%%%
