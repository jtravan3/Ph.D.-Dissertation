\section{Introduction}
\label{mls:introduction}

Now that the framework for transactional correctness has been developed (work published in \cite{ravan_ensuring_2020}) within the prediction-based solution there is potential to extend the reputation score of the prediction-based scheduler to a categorization for multi-level secure database. The added attribute to the reputation score would include the security classification and allow for a much more robust decision-model. This portion of the overall solution would focus on multi-level secure database systems and covert timing channels. By providing an additional attribute to the existing framework we can therefore extend our reputation score to include security levels. This allows us to provide a cover story for the timing difference of transactions with differing security classifications. 

The problem with multi-level secure databases is the possibility that there could be a covert channel allowing unauthorized access. The covert channel would provided the ability for a transaction of a lower security classification to access resources designed for a higher security classification. Existing research provides possible solutions but many of the solutions starve transactions of higher security classifications from gaining access to the resources needed (see Section \ref{mls:related_work}). With that cover story in mind we can then provide a solution that elevates high security transactions would the presence of a covert timing channel. With the prediction-based solution provided in Chapter \ref{chap:prediction_based_scheduler} and the reputation score provided in Chapter \ref{chap:dynamic_reputation}, we can provide a cover story that determines locking priority based on the reputation of the transaction as a whole. This prevents the starvation of higher security classification transactions. By taking into consideration the security classification as a metric to calculate the reputation of the transaction, we also prevent the presence of covert channels. 