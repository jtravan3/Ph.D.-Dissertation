\section{Related Work}
\label{mls:related_work}

Multi-level secure databases are a huge area of research due to the security concerns that can arise within these databases and the consequences if a vulnerability is exposed. The consequences have been so great that many users of multiple security classifications use multiple databases with duplicated common resources to prevent any security vulnerability from happening \footnote{This implementation is commonly found in DoD database systems and their Security Technical Implementation Guides (STIGs) \hyperlink{https://public.cyber.mil/stigs/}{https://public.cyber.mil/stigs/}}. However, researchers understand the benefits of having a secure solution contained in a single database system with multiple security levels.

Jajodia et. al. is one of the main motivations for our work (see \cite{jajodia_fair_1998}). In this publication we see a new locking protocol presented specifically for multi-level secure databases that prevents the starvation of lower security classification transactions. In this work, the researchers analyzed the existing two-phase locking protocol with additional policy additions to increase performance. In their analysis they discovered in order to increase performance and prevent starvation by increasing the fairness of all transactions, they needed to restrict the number of low security transactions executing. One quote from the work that motivates our work is,

\begin{displayquote}
"Several concurrency control algorithms that are free from covert channels have been proposed in the literature. Most of these algorithms prevent covert timing channels by ensuring that transactions at lower security levels are never delayed by the actions of a transaction at a higher security level. This can be accomplished by providing a higher priority to low transactions whenever a data conflict occurs between a high transaction and a low transaction."
\end{displayquote}

The prediction-based solution established in Chapter \ref{chap:prediction_based_scheduler} provides a solution to elevate or demote transactions based on transactional attributes that are deemed necessary for the transaction's reputation score. As a part of the reputation score, the security level in which a transaction resides can be a part of the transaction's attributes necessary for ranking.

Other, more recent, works that have been of influence for this solution involve \cite{mahmoud_encryption_2019} by Mahmoud and Alqumboz, \cite{sun_access_2011} by Ying-Guan Sun, \cite{hedayati_evaluation_2010} by Hedayati et. al., \cite{shanwal_secure_2013} by Shanwal and Kumar, \cite{sapra_development_2014} by Sapra et. al., \cite{kaur_performance_2004} by Kaur, N. et. al., \cite{costich_analysis_1991} by Costich, O.L. et. al., \cite{kaur_feedback_2007} by Kaur, N. et. al, \cite{keefe_database_1993} by Keefe T.F. et. al, and \cite{david_secure_1993} by David N. et. al. All of which have been built up on the work of the Bell–LaPadula Model, Biba Integrity Model, and lattice based security model (LBAC) (work referenced in \cite{bell_secure_1973}, \cite{biba_integrity_1977}, \& \cite{denning_lattice_1976}). An overview of multilevel secure databases and transaction processing can be found in Atluri et. al. (\cite{atluri_mls}).