\section{Transaction Quality Measure}
\label{mls:tqm}

In this section we discuss the potential future work for implementing a security classification within the prediction-based scheduler (outlined in Chapters \ref{chap:prediction_based_scheduler} and \ref{chap:dynamic_reputation}) that would address the issues found specifically in \gls{mls} databases. 

Within the prediction-based solution, there are currently four attributes that allow for a reputation score to be calculated among transactions. Those four attributes are commit ranking, efficiency ranking, user ranking and system ranking (see Definitions \ref{def:commit_ranking}, \ref{def:efficiency_ranking}, \ref{def:user_ranking}, and \ref{def:system_abort_ranking} in Chapter \ref{chap:dynamic_reputation}). This allows for the formation of a reputation score to be calculated for each transaction in the system. Within these reputation scores, there is a dominance structure that causes transactions to be prioritized depending on if dominance can be established (outlined in Definitions \ref{def:strong_dominance} and \ref{def:weak_dominance}). With all of these components in place, we have a foundation for efficient transaction categorization within \gls{mls} databases.

A potential future state of the system is to use the existing reputation score within the existing prediction-based solution alongside a security label to create a two-tuple Transaction Quality Measure (TQM). This will involve an update dominance structure to ensure the absence of covert channels and also prevent starvation of higher security transactions within the system. The solution would allow for a better decision making model for which transactions are aborted and rescheduled. The new Transaction Quality Measure would take security classification into account, but it would not allow the security classification to be the only dictating factor. Extending the prediction-based solution would allow the other four attributes to be included within the decision process. Figure \ref{fig:existing_reputation_score} is a representation of the existing reputation score defined in Chapter \ref{chap:dynamic_reputation}. Figure \ref{fig:mls_strong_dominance} shows the strong dominance structure of Transaction Quality Measures where $SL$ is the security label and $PV$ is the performance vector representing the existing reputation score. Figure \ref{fig:mls_weak_dominance} shows the weak dominance structure.

\begin{figure}[h]
\captionsetup{justification=centering}
\centering % used for centering Figure

\[\textrm{$RS_{T_{i}}$ = $<w_{i}^{1}\times CR_{T_{i}},w_{i}^{2}\times ER_{T_{i}},w_{i}^{3}\times UR_{i},w_{i}^{4}\times SR_{T_{i}}>$}\]

\caption{Current Reputation Score} % title of the Figure
\label{fig:existing_reputation_score} % label to refer figure in text

\end{figure}

\begin{figure}[h]
\captionsetup{justification=centering}
\centering % used for centering Figure

\[\textrm{
$TQM_{1}(SL_{1},PV_{1}) \geq TQM_{2}(SL_{2},PV_{2})$ iff $SL_{1} \leq SL_{2}$ and $PV_{1} \geq PV_{2}$}\]

\caption{MLS Strong Dominance} % title of the Figure
\label{fig:mls_strong_dominance} % label to refer figure in text

\end{figure}

\begin{figure}[h]
\captionsetup{justification=centering}
\centering % used for centering Figure

\[\textrm{
$TQM_{1}(SL_{1},PV_{1}) \geq TQM_{2}(SL_{2},PV_{2})$ iff $SL_{1} \geq SL_{2}$ and $PV_{1} > PV_{2}$}\]

\caption{MLS Weak Dominance} % title of the Figure
\label{fig:mls_weak_dominance} % label to refer figure in text

\end{figure}

In order to prevent covert timing channels within multi-level secure database, the future solution would leverage the timing delay of the existing prediction-based solution to be used as a "cover story" for the timing difference between transactions of differing security levels. The cover story would allow for transactions to be aborted for conflicting transactions without introducing a covert timing channel for unauthorized disclosure of high security resources.

In summary, there are two well-known problems within multi-level secure databases. The first problem is the existence timing covert channels when transactions of multiple security levels are accessing a common resource. The presence or absence of a time delay provides the indication of high security resources that are available. The second problem, is brought on by the solution to multi-level secure databases. A solution to prevent against covert channels is to abort transactions of a higher security classification so that the time delay does not exist. However, this then causes these transactions to suffer from starvation and will never be executed. The solution presented in this section will address both problems and provide a way forward for more granular decision-making within multi-level secure database systems.

With the prediction-based scheduler in place and a solution for dynamic reputation of transactions, the possibilities for extension within multi-level secure databases is then feasible . Chapter \ref{chap:prediction_based_scheduler} presented the solution and operated under the assumption that the reputation of the transactions were already established. Chapter \ref{chap:dynamic_reputation} focuses on exactly how the transactions establish their reputation and also dynamically increase or decrease their reputation. With this work in place, extending the prediction-based system to multi-level secure databases is now possible.