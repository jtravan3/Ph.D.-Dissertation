\section{Additional Future Work}

In this section, we want to outline the future work opportunities of the prediction-based scheduler and how the work can be expanded upon. The work mentioned in this section is not meant to be included in this dissertation, but rather listing outstanding opportunities for the current and proposed work to continue forward.

\subsection{Snapshot Isolation}
Another potential for future work is the ability to perform snapshot isolation within the different categorizations of transactions. This extension will be for both malicious and lower priority transactions that affect the majority of well-performing transactions. In this work we'll use snapshot isolation to execute certain categorizations of transactions on snapshots of the database in order to prevent the effects of low-performing transactions from affecting all transactions. Once the outcome of a transaction has been determined then the snapshot can either be discarded or merged.

\subsection{Prediction-based Scheduling within Linked Databases}
An additional extension is the issue of efficient concurrency control within linked database environments. Currently the Prediction-based solution addresses efficient concurrent transactions within a web-service environment that are contained within a single cluster. This particular area of research will address the problem through the lens of the Prediction-based solution. The difficulty of the problem within this work is adapting the existing framework of correctness built within the Prediction-based solution so that it will scale to linked database systems while preserving its existing capabilities. Figure \ref{fig:system_model_linked_databases} shows the system model for the Prediction-based solution within linked database systems.

\begin{figure}[h]
\captionsetup{justification=centering}
\centering
\includegraphics[width=\textwidth]{images/LinkedDatabase_SystemModel}
\caption{Prediction-based Scheduler within Linked Databases}
\label{fig:system_model_linked_databases}
\end{figure}

\subsection{PostgreSQL \& MySQL}
\label{conclusion:posgressql}
Two very commonly used databases within enterprise applications are \href{https://www.postgresql.org/}{PostgresSQL} and \href{https://www.mysql.com/}{MySQL}. Both of which are open-source relational databases where their code is available to the public for modification and contribution. Open-source applications tend to have very difficult review process which allows for quality code and reliable software. Both of these database management systems have provided their code on Github so that the community can contribute features, bug fixes, and enhancements accordingly. The code for PostgresSQL is located at \href{https://github.com/postgres/postgres}{https://github.com/postgres/postgres} and the code for MySQL is located at \href{https://github.com/mysql/mysql-server}{https://github.com/mysql/mysql-server}.

One opportunity for future work would be to fork one or both of these repositories on a controlled system and implement the algorithms of the prediction-based scheduler within the database management system itself. Currently, the prediction-based scheduler has been proven theoretically in a test environment using an in-memory database. This work has proven the viability of the solution and the consistency that it provides. By placing the prediction-based solution in the database management system itself, it would provide a beautiful marriage of academia and industry coming together for a common solution. The initial goal would be to get the algorithms working on a mirrored fork initially in a controlled test environment, then moving that solution to a clustered environment to ensure scalability, and eventually providing an official pull request of the prediction-based scheduler to the code maintainers of both systems so the solution would then be available to the general public in future releases. This future work provides a direct road map from academic theory to impacting the global industry for the benefit of the masses.