\section{Remaining Work}
\label{rep:remaining_work}

In this section we discuss the proposed work for implementing dynamic reputations within the prediction-based scheduler. This section contains the proposal itself. The background information, problem definition, and existing research for the problem is outlined in Chapter \ref{chap:dynamic_reputation}.

First, we will work towards building a dynamic reputation among transactions. Currently the transaction categorizations are assumed in order make way for the feasibility of the solution as a whole. This work will focus on the actual categorization and reputation management of the transactions. The work needed is a system that properly records metrics, categorizes the transactions, and dynamically can reassign categories if the metrics within the system change.

The first part of this work will be creating and establishing classes of transactions rather than individual transactions. A flyweight is needed in order to extract the intrinsic transactional data from the extrinsic data. This will then reveal the classes of transactions that a reputation can be associated with.

The second and final part of this work will be leveraging reputation management solutions in order to appropriately track the reputation of transaction classes. At that point, the reputation of a transaction class will dictate the transactional category (see Definition \ref{transaction_categories}) that the transaction is placed in. The dynamic reputation then translates to a dynamic promotion and demotion of the category of a transaction and therefore the locking mechanisms associated.

We propose researching the current academic environment for what has been accomplished already, providing a solution based on the current research, verifying the accuracy through proofs and use-case scenarios, and working toward a viable prototype with initial results. Since the bulk of the theoretical and simulation results have been accomplished in Chapter \ref{chap:prediction_based_scheduler} through the publication of \cite{ravan_ensuring_2020} we propose that this work consist of a submission to a conference.