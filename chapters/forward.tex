The following dissertation took many years of sacrifice and dedication to complete. It took hours of meetings, proofreading, reading, rereading, editing, starting over, and simply scratching our heads sometimes until solutions presented themselves. But before this dissertation had even started to take shape it started with me in a small blue collar town in South Carolina where, in my eyes, Computer Science was a luxury and not a necessity. 

I grew up in Woodruff, South Carolina. Woodruff is a small blue collar town in Upstate South Carolina with a population of a little over 4,000 residents. I grew up working hard and enjoying the little things in life like most of my family. I was an outdoorsman and loved hunting and fishing like most of my friends. On rainy days me and my friends would play video games inside to pass the time. It started with playing simple single user games and then progressed into multi-user online role playing games where groups of us would connect at night to accomplish certain missions. It was then when I began to find a new love that could overcome the outdoors. I couldn't decide if I enjoyed the company of friends playing together or the technology that made the game possible more at the time but little did I know that a basic love for technology and the things it made possible would lead me into a career path that could potentially further that technology for generations to come.

My senior year of high school comes and just like all of my other friends we're deciding what college we plan to attend and more importantly what we plan to study. After my love of video games and the technology that made it possible I began to appreciate technology in the general sense. No matter if it was a new computer that was released or a new circular saw for woodworking, I could begin to see the technology that was needed to make these advancements possible. With technology as my primary love then Computer Science became more and more apparent as my focus. Much to my surprise, studying Computer Science has very little to do with video games or circular saws. 

There were many times that I never thought I was going to make it. Taking 18-20 hours a semester, studying on the weekends, taking summer classes. Some days I never thought it would end but I was able to push through. In 2011 I graduated The Citadel with a B.S. in Computer Science. Three years later I was able to complete my M.S. in Software Engineering.

During my junior year I was able to start working as a software engineering intern for a company local to Charleston, SC. Ever since then I have worked as a software engineer in some capacity for multiple companies. Working as a software engineer while also being involved in academia was the best pairing I could give myself going forward. I could see both sides of the coin in how technology really did further multiple advancements. I could see the theory of software design and architecture while using the newest software framework that implemented those architectural concepts. I could intelligently defend why it was so important that software engineers understand the underlying theoretical concepts rather than simply importing a framework and continuing on. It allowed me to see that academia had a place and it wasn't in a silo with other researchers but it was meant to be paired with society. I couldn't stop.

In August of 2014 I started my Ph.D. in Computer Science. Much of the same struggles I faced in my previous years in academia would continue but, once again, I was able to overcome. When you read the following work don't make the mistake of thinking this work was done in a silo of academia. What you have before you is a culmination of over 12 years of experience of computer science education and industry software engineering. The work before you provides a solid theoretical contribution but is also backed by experimentation results and prototypes that were built using the latest technologies and processes used by hundreds of companies for their applications. I never understood why there were resources built for academia proof of concepts that were designed to address an industry problem. This work marries the two concepts together. In this work academia crosses the finish line with industry driving the car.