\section{Conclusion}
\label{sec:conclusion}

% In closing we conclude that this extension of the prediction-based scheduling solution involves the identification of two unique problems: identifying a class of transactions and maintaining a dynamic reputation for that class of transactions. Identifying a class of transactions allows PBS the ability to assign a categorization to transactions with common intrinsic attributes. Separating the intrinsic attributes from the extrinsic involves a flyweight modeling approach in order to properly model transaction classes. 

In closing we conclude that the dynamic reputation solution provides a system of greater reward and punishment than the previous prediction-based solution. The four category system of the prediction-based solution doesn't contain the level of granularity needed to establish dominance in conflicting situations. We also conclude that the overhead of the dynamic reputation solution is comparable to the previous prediction-based and \gls{2pl} solutions therefore establishing a resource management solution with a feasible overhead and consistent scheduling. By analyzing our findings from the experimentation we can conclude that execution environments with workloads of greater than 20\% conflicting transactions would benefit from the granularity of the dynamic reputation scheduler to provided consistency and efficient scheduling.