\section{Related Work}
\label{sec:related_work}

There are many resources available for dynamic reputation management that shaped our understanding of the challenges of maintaining reputation of nodes/services. Many of the reputation systems were centered around maintaining a reputation in a distributed or decentralized system. These resources were \cite{clark_dynamic_2017}, \cite{de_paola_reputation_2008}, and  \cite{hu_reputation_2010}. While this helped our understand of the challenges of reputation management, our system would not experience the challenges associated with a decentralized trust management solution since the service and the database are both contained within the prediction-based scheduler itself. Other reputation management solutions were focused on the reputation or trustworthiness of web pages. Those solutions were presented in \cite{melnikov_towards_2018}, \cite{wang_research_2008}, and \cite{zhang_how_2012}. These were also helpful but didn't address the reputation of database transactions directly which is the focus of the reputation of the prediction-based solution. The majority of reputation management systems are provided with the context of P2P or ad-hoc mobile networking in mind. Resources focused on dynamic reputations within networking environments are \cite{chiejina_dynamic_2014}, \cite{de_paola_reputation_2008}, \cite{hu_reputation_2010}, and  \cite{sun_dynamic_2019}.

We also looked at various multi-agent reputation systems to understand how their reputation management systems were leveraged. Reputation management within multi-agent systems, web service selection, and e-commerce has been researched more in depth than reputation within database systems. This gave us a different perspective into reputation management that we could transfer to our database solution. The solutions we reviewed were "Multiagent reputation management to achieve robust software using redundancy" by Rajesh Turlapati (see \cite{rajesh_turlapati_multiagent_2005}), "Multiagent System for Reputation–based Web Services Selection" by Wang et. al. (see \cite{wang_multiagent_2006}), and "Swarm Intelligence Based Reputation Model for Open Multi Agent Systems" by Mahmood et. al. (see \cite{mahmood_swarm_2006}).

Understanding how the user plays a role with the outcome of the transaction is a key piece of our solution. Looking at the work presented by Lomet et. al. (\cite{lomet2006recovery}) motivated our inclusion of the User Ranking (see Definition \ref{def:user_ranking}) in our solution.

To better understand data lineage and how it can be used as a tool for building transactional reputations we looked at our previous research of data provenance and malicious transactions in Theppatorn et. al. (\cite{theppatorn_2021}).

A great deal of the existing work related to concurrency control (e.g., \cite{Alrifai_Distributed_Managment}, \cite{Fekete_RAMP}, \cite{dai_qos-driven_2009}, \cite{WSCO}, \cite{ferreira_transactional_2012}, \cite{WSBA}, \cite{zhengdong_gao_combining_2005}, \cite{Fekete_IsolationSupport}, \cite{Fekete_Promises}, \cite{Eunhee_PredictionBasedCC}, \cite{kang-woo_lee_consistency_2000}, \cite{WSAT}, \cite{olmsted_long_2015}, and \cite{Riegen_RuleBased}) influenced our motivation for this work.

After studying the current environment of dynamic reputation systems, we believe this is a great opportunity to provide a dynamic reputation management solution particularly focused at database transactions within a web service environment.