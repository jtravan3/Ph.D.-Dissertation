\section{Motivations}
By looking at a practical use case we can more clearly see the issue and the need for a solution that ensures consistency. In Figures \ref{fig:e_com_ticket} \& \ref{fig:bp_env}, we see five web services executing on three different database instances. The first four web services  create a common business process created by the Business Process Execution Language (BPEL) \cite{BPEL}. The web services are: $WS_{1}$ (decrement inventory by product ID), $WS_{2}$ (process payment), $WS_{3}$ (add order by user ID), and $WS_{4}$ (delete user payment info). The goal of the process is to allow a customer to purchase a product from an e-Commerce site. $WS_{5}$ (delete user payment info) and $WS_{3}$ execute within the same database instance. With the relaxed properties in the web service context, concurrent executions of $WS_{5}$ and $WS_{3}$ could cause an inconsistent state on $Node_{3}$. This would then cause a cascading rollback to execute and revert the committed operations of $WS_{1}$ and $WS_{2}$. Existing research shows that many solutions have been presented in the past to address this issue (e.g., \cite{Fekete_SnapshotIso}, \cite{Alrifai_Distributed_Managment}, \cite{Fekete_RAMP}, \cite{Fekete_IsolationSupport}, \cite{Jacobi_Locking}, and \cite{Fekete_Promises}). 

The most influential research that inspired the prediction-based solution was the Promises Model. The Promises model presented by Alan Fekete et al. (e.g., \cite{Fekete_IsolationSupport} and \cite{Fekete_Promises}) is an elegant solution that "promises" a particular transaction that the requested resource will be available while allowing concurrent transactions to still execute on that resource. The Promises solution is robust in that it allows the "strengthening" or "weakening" of promises after they have already been made. This allows existing promises on resources to be modified without breaking the existing promise entirely. However, the solution introduces backwards compatibility issues along with a potential bottleneck at the occurrence of registering a promise for a particular transaction.

However, none of the existing work improves currency control based on the performance of the transactions. That is the likelihood that the transaction will commit and the computational cost of the transaction. In our work we propose improvements for concurrency control using these performance characteristics. 

