\section{Conclusion}
\label{mls:conclusion}
In summary, there are two well-known problems within multi-level secure databases. The first problem is the existence timing covert channels when transactions of multiple security levels are accessing a common resource. The presence or absence of a time delay provides the indication of high security resources that are available. The second problem, is brought on by the solution to multi-level secure databases. A solution to prevent against covert channels is to abort transactions of a higher security classification so that the time delay does not exist. However, this then causes these transactions to suffer from starvation and will never be executed. The solution (proposed in Section \ref{mls:remaining_work} will address both problems and provide a way forward for more granular decision-making within multi-level secure database systems.

\subsection{Concluding Remarks}
With the prediction-based scheduler in place and a solution for multi-level secure databases, we can now focus on the reputation of the transactions themselves. In the next chapter, this is where we will place our focus. Chapter \ref{chap:prediction_based_scheduler} presented the solution and operated under the assumption that the reputation of the transactions were already established. Chapter \ref{chap:multi_level_security} focuses on the extension for multi-level secure databases while Chapter \ref{chap:dynamic_reputation} focuses on exactly how the transactions establish their reputation and also dynamically increase or decrease their reputation.